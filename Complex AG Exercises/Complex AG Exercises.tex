\documentclass{article}
\usepackage{./cls/standard}

\begin{document}
\title{Complex Algebraic Geometry Exercises}
\author{Matei Ionita}

\maketitle

The following are solutions to selected exercises in \cite{Huy}, \cite{Mir} and \cite{Voi1}. These exercises are
intended as practice for a Complex Algebraic Geometry course taught at the University of Pennsylvania
by Ron Donagi.

\section{Miranda}
\subsection*{Problem II.4.G in \cite{Mir}}
\textit{Let} $f(z) = z^3/(1-z^2)$ \textit{considered as a meromorphic function on the Riemann sphere}
$\C_{\infty}$. \textit{Find all points} $p$ \textit{such that} $\ord_p(f) \neq 0$. \textit{Consider the
associated map} $F: \C_{\infty} \to \C_{\infty}$. \textit{Show that} $F$ \textit{has degree 3 as a
holomorphic map, and find all of its ramification and branch points. Verify Hurwitz's formula for this map.}
\begin{proof}
At $z=0$ the expression we are given for $f(z)$ already makes evident that its Laurent expansion has
$a_3$ as the lowest nonzero coefficient, hence $\ord_0(f) = 3$. Shifting to coordinates $w = z \pm 1$, we
find that the Laurent expansion around $z = \pm 1$ has $a_{-1}$ as the lowest nonzero coefficient,
hence $\ord_{\pm 1}(f) = -1$. Performing a similar shift $w = z - a$ for $a \neq 0, \pm 1$ shows that
$\ord_{a}(f) = 0$. The only thing left to check is $\ord_{\infty}(f)$, for which we use the coordinate
$w = z^{-1}$. Then:
\[	f(w) = \frac{1}{w^3(1-w^{-2})} = \frac{1}{w(1-w^2)}	\]
Hence $f$ has a pole of order one at $w=0$, which means that $\ord_{\infty}(f) = -1$.

Using Lemma 4.7, we can relate the multiplicity of $F$ at various points to the orders of $f$ computed
above. In particular, $\mult_0(F) = 3$, and $\mult_{\pm 1}(F) = \mult_{\infty} = 1$. For other
$p \in \PP^1$, we could apply part (c) of Lemma 4.7, but that leads to cumbersome computations. We use
instead Lemma 4.4. On $\PP^1 - \{\infty\}$, $z$ is a coordinate function, and:
\[	F'(z) = \frac{z^2(3-z^2)}{(1-z^2)^2}.	\]
We see again that $\mult_0(F) = 1+2 = 3$, but we also find that $\mult_{\pm\sqrt{3}}(F) = 1+1=2$,
which is not obvious from the expression of $F(z)$. Hence the ramification points are:
\begin{itemize}
\item $z=0$, with multiplicity 3;
\item $z= \pm \sqrt{3}$, with multiplicity 2.
\end{itemize}
The branch points are:
\begin{itemize}
\item 0, with 1 preimage;
\item $\pm 3\sqrt{3}/2$, each with 2 preimages.
\end{itemize}
To obtain the degree of $F$ we look at $F^{-1}(0) = \{0\}$; using Proposition 4.8 we obtain 
$\deg (F) = 1\cdot 3 = 3$. With this data, the Hurewitz formula gives:
\[	-2 = -2 \cdot 3 + [2+1+1],	\]
so everything is consistent.
\end{proof}


\subsection*{Problem II.4.I in \cite{Mir}}
\textit{Let} $F:X\to Y$ \text{be a nonconstant holomorphic map between compact Riemann surfaces.}
\begin{enumerate}
\item \textit{Show that if} $Y \cong \PP^1$, \textit{and} $F$ \textit{has degree at least two, then}
$F$ \textit{must be ramified.}
\item \textit{Show that if} $X$ \textit{and} $Y$ \textit{both have genus one, then} $F$ \textit{is
unramified.}
\item \textit{Show that} $g(Y)\leq g(X)$.
\item \textit{Show that if} $g(Y) = g(X) \geq 2$, \textit{then} $F$ \textit{is an isomorphism.}
\end{enumerate}
\begin{proof}
Recall the Hurwitz formula for a holomorphic map between Riemann surfaces:
\[	2g(X) - 2 = \deg(F)\big(2g(Y) - 2\big) + \sum_{p\in X} [\mult_p(F) - 1].	\]
\begin{enumerate}
\item The Hurwitz formula when $g(Y) = 0$ gives:
\begin{align*}
\sum_{p\in X} [\mult_p(F) - 1] &= 2g(X) - 2 + 2\deg(F) \geq 2 + 2g(X) > 0.
\end{align*}
\item When $g(X) = g(Y) = 1$, the Hurwitz formula reduces to $\sum_{p\in X} [\mult_p(F) - 1] = 0$.
Since each term is nonnegative, $\mult_p(F) = 1$ for every $p$.
\item Rewrite the Hurwitz formula as:
\[	2\big(g(X) - g(Y)\big) = \big(\deg(F) - 1\big)(2g(Y) - 2) + \sum_{p\in X} [\mult_p(F) - 1].	\]
If $g(Y) = 0$, then $g(Y) \leq g(X)$ is automatic. So assume that $g(Y) \neq 0$. Then the RHS is
$\geq 0$, from which it follows again that $g(Y) \leq g(X)$.
\item For $g(X) = g(X) \geq 2$, the formula reduces to:
\[	(2g(X) - 2)\big(1-\deg(F)\big) = \sum_{p\in X} [\mult_p(F) - 1].	\]
The RHS is $\geq 0$ and $2g(X) - 2 > 0$, hence $\deg(F) \leq 1$. The only possibility is $\deg(F) = 1$,
hence $F$ is an isomorphism.
\end{enumerate}
\end{proof}

\subsection*{Problem II.4.J in \cite{Mir}}
\textit{Let} $X$ \textit{be the projective plane curve of degree} $d$ \textit{defined by the homogenous
polynomial} $F(x,y,z) = x^d + y^d + z^d$. \textit{This curve is called the Fermat curve of degree}
$d$. \textit{Let} $\pi:X \to \PP^1$ \textit{be given by} $\pi[x:y:z] = [x:y]$.
\begin{enumerate}
\item \textit{Check that the Fermat curve is smooth.}
\item \textit{Show that} $\pi$ \textit{is a well-defined holomorphic map of degree} $d$.
\item \textit{Find all ramification and branch points of} $\pi$.
\item \textit{Use the Hurwitz formula to compute the genus of the Fermat curve. You should get:}
\[	g(X) = \frac{(d-1)(d-2)}{2}.	\]
\end{enumerate}
\begin{proof}
\begin{enumerate}
\item The derivative of $F$ is:
\[	DF = \left[ \begin{array} {ccc} dx^{d-1} & dy^{d-1} & dz^{d-1} \end{array} \right].	\]
Hence $X$ is singular only if $[x:y:z] = [0:0:0] \not \in \PP^2$.
\item To see that $\pi$ is well-defined, note that $[x:y] = [0:0]$ together with $F(x,y,z) = 0$
implies that $[x:y:z] = [0:0:0] \not \in \PP^2$. Hence $\pi$ always maps to $\PP^1$; it is also
clear that $\pi$ is independent of the representative chosen for $[x:y:z]$.

To see that $\pi$ is holomorphic, we look at its coordinate representation in the standard charts
for $\PP^2$ and $\PP^1$. Consider the chart $(U_0,\phi_0)$ on $\PP^2$ given by $U_0 = \{[x:y:z] \in X 
| x\neq 0\}$ and:
\begin{align*}
\phi_0 : U_0 &\to \C^2 \\
[x:y:z] &\mapsto (w_1, w_2) := \left(\frac{y}{x}, \frac{z}{x}\right).
\end{align*}
Moreover, consider the chart $(V_0, \psi_0)$ on $\PP^1$ given by $V_0 = \{[x:y] \in \PP^1 | x \neq 0\}$
and:
\begin{align*}
\psi_0 : V_0 &\to \C \\
[x:y] &\mapsto w := \frac{y}{x}.
\end{align*}
Then $\pi(U_0) \subset V_0$ and $\psi_0 \circ \pi \circ \phi_0^{-1}(w_1, w_2) = w_1$, which is holomorphic.
One can check in a similar fashion that the other coordinate representations of $\pi$ are holomorphic as
well.

We compute the degree of $\pi$ by looking at the preimage of $p = [1:0] \in \PP^1$. $\pi^{-1}(p) =
\{ [1:0:\xi_i] \}$, where $\{\xi_i\}_{i=1}^d$ are the $d^{\text{th}}$ roots of -1. $\pi$ is unramified
at $[1:0:\xi_i]$, using Lemma 4.6 and the fact that:
\[	\p_{w_2}(F\circ \phi_0^{-1}) = \p_{w_2}(1 + w_1^d + w_2^d) = dw_2^{d-1}	\]
does not vanish at $[1:0:\xi_i]$. Hence $\deg F = \sum_{q \in \pi^{-1}(p)} 1 = d$.
\item We claim that there are no ramified points outside $U_0 = \{[x:y:z] \in X | x\neq 0\}$. Indeed,
all $p \in X - U_0$ map to $[0:1] \in \PP^1$, the preimages of which are $[0:1:\xi_i]$. In particular,
$[0:1]$ has $d$ preimages, so it is not a branched point.

We can then work in $U_0$ and use the coordinate representation $F\circ \phi_0^{-1}(w_1,w_2) = 
1+ w_1^d + w_2^d$ of $F$. Since $\psi_0 \circ \pi \circ \phi_0^{-1}(w_1, w_2) = w_1$, Lemma 4.6 implies
that the ramification points of $\pi$ are the solutions to $\p_{w_2} (F\circ \phi_0^{-1}) = 0$ in $X$.
These are $[1:\xi_i:0]$; there are $d$ of them and each has multiplicity $d$. The corresponding
branch points are $[1:\xi_i]$.
\item Using the above information in the Hurwitz formula results in:
\[	2g-2 = -2d + d(d-1).	\]
Hence $g = \frac{1}{2}(d-1)(d-2)$.
\end{enumerate}
\end{proof}


\subsection*{Problem IV.1.A in \cite{Mir}}
\textit{Let} $X$ \textit{be the Riemann sphere, with local coordinate} $z$ \textit{in one chart
and} $w=1/z$ \textit{in the other. Let} $\omega$ \textit{be a meromorphic 1-form on} $X$. \textit{Show
that if} $\omega = f(z) dz$ \textit{in the coordinate} $z$, \textit{then} $f$ \textit{must be a rational
function of} $z$. \textit{Show further that there are no nonzero holomorphic 1-forms on} $X$. \textit{What
are the zeros and poles, and the respective orders, of the 1-forms} $dz$ \textit{and} $dz/z$?
\begin{proof}
In the coordinate chart around $z=\infty$, $\omega$ is given by $-f(w^{-1})w^{-2} dw$. (We know that
$\omega$ transforms this way on the overlap of the two charts; the uniqueness of analytic continuation
then guarantees that the extension to $z = \infty$ is unique.) Since $\omega$
is meromorphic, its component functions in each chart must be meromorphic. Hence $f(z), f(w^{-1})$ are
both meromorphic on $\C$, which means that they are the coordinate representations of a meromorphic
function on $X$. Using Theorem 2.1 in Chapter II, $f(z)$ is a rational function.

If $\omega$ is holomorphic, then particular $f$ is holomorphic on each chart, so it defines a holomorphic
function on $X$. Since $X$ is compact, $f$ is constant. However, this is incompatible with the transformation
law $f(z) \mapsto -w^{-2}f(w^{-1})$ that $F$ satisfies. We conclude that $X$ admits no holomorphic 1-forms.

$\omega_1 = dz$ is given by the expression $-w^{-2} dw$ on the chart around $z=\infty$. Hence $\omega_1$
has no zeros on $X$, and a pole of order 2 at $z=\infty$. $\omega_2 = z^{-1}dz$ is given by the expression
$-w^{-1}dw$ on the chart around $z=\infty$. Hence $\omega_2$ has no zeros on $X$, and poles of order 1
at $z = 0$ and $z=\infty$.
\end{proof}

\subsection*{Problem IV.1.C in \cite{Mir}}
\textit{Let} $X$ \textit{be a smooth affine plane curve defined by} $f(u,v) = 0$. \textit{Show that}
$du$ \textit{and} $dv$ \textit{define holomorphic 1-forms on} $X$, \textit{as do} $p(u,v) du$ \textit{and}
$p(u,v)dv$ \textit{for any polynomial} $p(u,v)$. \textit{Show that if} $r(u,v)$ \textit{is any rational
function, then} $r(u,v) du$ \textit{and} $r(u,v)dv$ \textit{are meromorphic 1-forms on} $X$. \textit{Show 
that} $(\p f/\p u) du= - (\p f/\p v) dv$ \textit{as holomorphic 1-forms on} $X$.
\begin{proof}

\end{proof}




\section{Huybrechts, Chapter 1}

\subsection*{Exercise 1.1.1 in \cite{Huy}}
\textit{Show that every holomorphic map}
\[	f : \C \to \HH : = \{z | \Imag (z) > 0	\}	\]
\textit{is constant.}
\begin{proof}
Using the Riemann mapping theorem, there exists a biholomorphism $\phi :\HH \to \D$. Then
$\phi \circ f : \C \to \D$ is holomorphic, or in other words $\phi \circ f$ is entire and bounded.
It follows from the Liouville theorem that $\phi \circ f$ is constant. Hence $f = \phi^{-1} \circ
(\phi \circ f)$ is also constant.
\end{proof}

\subsection*{Exercise 2.1.2 in \cite{Huy}}
\textit{Show that} $\C^n$ \textit{does not have any compact submanifolds of positive dimension. (This
is in contrast to the real situation, where any real manifold, compact or not, can be realized as
a submanifold of some} $\R^n$\textit{.)}
\begin{proof}
Let $X$ be any compact complex submanifold of $\C^n$, and let $\iota:X \to \C^n$ be the inclusion.
Let $f:\C^n \to \C$ be any holomorphic function. Then $f \circ \iota \in \Gamma(X,\mathcal{O}_X)$. Since
$X$ is compact, $\Gamma(X,\mathcal{O}_X) \cong \C^k$, where $k$ is the number of connected components of
$X$. Hence $f\circ \iota$ is locally constant. If $\dim X >0$, it follows that $f$ is constant on a sequence
of distinct points which converge in $\C^n$. Using the uniqueness of analytic continuation, $f$ is constant on $\C^n$.
This is a contradiction, since $\C^n$ admits nonconstant holomorphic functions. Then it must be the case that
$\dim X = 0$.
\end{proof}


\subsection*{Exercise 2.1.3 in \cite{Huy}}
\textit{Determine the algebraic dimension of the following manifolds:} $\PP^1, \PP^n, \C/(\Z + i\Z)$.
\textit{For the latter, you might need to recall some basic facts on the Weierstrass} $\wp$
\textit{function. How big is the function field of} $\C$?
\begin{proof}
We use the standard open covering for $\PP^n$, i.e. $U_i = \{(z_0:\dots:z_n)|z_i\neq 0\} \subset \PP^n$,
with coordinate charts:
\begin{align*}
\phi_i : U_i &\to \C^n, \\
(z_0:\dots:z_n) &\mapsto \left(\frac{z_0}{z_i}, \dots, \frac{z_{i-1}}{z_i}, \frac{z_{i+1}}{z_i}, \dots,
\frac{z_n}{z_i} \right).
\end{align*}
We treat the case of $\PP^1$ separately, in order to develop an intuition for working with meromorphic
functions on $\PP^n$. Define the function:
\begin{align*}
f : \PP^1 &\to \C \\
(z_0:z_1) &\mapsto \frac{z_1}{z_0}.
\end{align*}
$f$ has coordinate representations $(f \circ \phi^{-1}_0)(w) = w$ and $(f\circ \phi^{-1}_1)(w) = \frac{1}{w}$.
It's easy to see that $f$ is holomorphic on $U_0$ and has a pole of order 1 at $(0:1)$, hence $f$ is meromorphic
on $\PP^1$. Moreover, rational functions of $f$ have coordinate representations which are rational functions
of $w$, so they are also meromorphic. We have proved that $\C(f) \subset K(\PP^1)$, which implies that 
$\trdeg_{\C} K(\PP^1) \geq 1$. Using Siegel's theorem and the fact that $\dim \PP^1 = 1$, we obtain the equality
$\trdeg_{\C} K(\PP^1) = 1$.

Moving on to $\PP^n$ for $n>1$, define the functions:
\begin{align*}
f_i : \PP^n &\to \C \\
(z_0:\dots:z_n) &\mapsto \frac{z_i}{z_0},
\end{align*}
for $1\leq i \leq n$. The coordinate representation of $f_i$ on the chart $(U_j, \phi_j)$ is:
\[	f_i\circ \phi_j^{-1}(w_0, \dots, \hat w_j, \dots w_n) =
\left\{ \begin{array} {c}
w_i \text{ for } j=0 \\
w_0^{-1}w_i \text{ for } j\neq 0
\end{array}\right.	\]
It follows that $f_i$, as well as rational functions in $\{f_i\}$, are meromorphic on $\PP^n$.
Moreover, $\{f_i\}$ are algebraically independent because their coordinate representations are.
Hence $\C(f_1, \dots, f_n) \subset K(\PP^n)$, and $\trdeg_{\C} K(\PP^n) \geq n$. Using Siegel's
theorem again, we obtain equality: $\trdeg_{\C} K(\PP^n) = n$.

For the case of $\C/(\Z + i\Z)$, we recall some results from the theory of doubly
periodic functions. Let $\Lambda = \Z + i\Z$ and $\Lambda^{\times} = \Lambda - \{0\}$. A meromorphic
function on $\C/(\Z + i\Z)$ is the same as a meromorphic function on $\C$ which is doubly periodic
with period $\Lambda$. The latter are generated as a field over $\C$ by the Weierstrass $\wp$-function:
\[ \wp(z) = \frac{1}{z^2} + \sum_{w\in \Lambda^*} \left[\frac{1}{(z+w)^2} - \frac{1}{w^2} \right] \]
and its derivative $\wp'$. Moreover, the two generators satisfy the equation:
\[	(\wp'(z))^2 = 4 (\wp(z))^3 - g_2 \wp(z) - g_3	,\]
where $g_2, g_3 \in \C$. Hence $K(\C/\Lambda)$ is the field of fractions of
\[	R/\fr{p}= \C[\wp, \wp']/(\wp'^2 - 4\wp^3 + g_2\wp + g_3).	\]
In particular, $\trdeg_{\C} K(\C/\Lambda) = 1$, since $\fr p$ is prime and $R/\fr{p}$ has Krull dimension 1.

Since $\C$ is not compact, Siegel's theorem does not impose any bound on $\trdeg_{\C} K(\C)$. And indeed,
the transcendence degree is infinite, which can be seen by considering the subfield of $K(\C)$ generated
by the holomorphic functions $\{\exp(t_i z)\}$, for $\{t_i\}_{i=1}^{\infty}$ complex. An algebraic relation
between $\{\exp(t_i z)\}$ over $\C$ gives an algebraic relation for $\{t_i\}_{i=1}^{\infty}$ over $\Q$. There
are $t_i$ for which the latter does not exist, because $\trdeg_{\Q} \C$ is infinite.
\end{proof}


\section{Huybrechts, Chapter 2}

\subsection*{Exercise 2.1.4 in \cite{Huy}}
\emph{Show that any holomorphic map from $\PP^1$ into a complex torus is constant. What about maps from
$\PP^n$ into a complex torus?}
\begin{proof}
Let $f$ be a holomorphic map $\PP^n \to \C^n / \Lambda$, where $\Lambda$ is a lattice of rank $2n$ in $\C^n$.
$\PP^n$ is the base of a fibration $S^1 \hookrightarrow S^{2n+1} \twoheadrightarrow \PP^n$. Using the associated
long exact sequence of homotopy groups, we see that $\pi_1(\PP^n) = 0$ for every $n$. Then $f$ has a lift
$\tilde f : \PP^n \to \C^n$ to the universal covering $\C^n$ of the torus $\C^n / \Lambda$. A priori $\tilde f$
is continuous, but not necessarily holomorphic.

For any $x \in \PP^n$, let $p = f(x)$ and $q = \tilde f(x)$. Then $\pi(q) = p$, where $\pi : \C^n \to \C^n /\Lambda$
is the holomorphic quotient map. $p$ has a neighborhood $U$ such that $\pi^{-1}(U)$ consists of disconnected
opens isomorphic to $U$; a choice of a connected component of $\pi^{-1}(U)$ gives a holomorphic chart $\phi$ around
$p$. Moreover, the coordinate representation of the quotient map is just a translation by an element $\omega
\in \Lambda$:
\[	\phi \circ \pi(q) = q + \omega. \]
Choose a holomorphic chart $(\psi, V)$ around $x\in \PP^n$ such that $f(V) \subset U$. The coordinate representation
of $\tilde f$ is $\tilde f \circ \psi^{-1}$, and we claim that this is a holomorphic map. Indeed:
\begin{align*}
\tilde f \circ \psi^{-1} (z) &= (\phi \circ \pi)^{-1} \circ (\phi \circ \pi) \circ (\tilde f \circ \psi^{-1}) (z) \\
&= \phi \circ (\pi \circ \tilde f) \circ \psi^{-1} (z) - \omega \\
&= \phi \circ f \circ \psi^{-1} (z) - \omega,
\end{align*}
which is holomorphic, since $\phi \circ f \circ \psi^{-1}$ is the coordinate representation of the holomorphic
map $f$. We conclude that $\tilde f$ is holomorphic.

But $\PP^n$ is compact, so $\tilde f$ must be constant. Hence $f = \pi \circ \tilde f$ is also constant.
\end{proof}


\subsection*{Exercise 2.1.5 in \cite{Huy}}
\emph{Consider the Hopf curve $X = \C^{\times} / \Z$, where $k \in \Z$ acts by $z \mapsto \lambda^k z$, for
$\lambda \in (0,1)$. Show that $X$ is isomorphic to an elliptic curve $\C/\Gamma$ and determine $\Gamma$
explicitly.}
\begin{proof}
Regard $\C^{\times}$ as $\C / \Z$, as per the following short exact sequence:
\[
\begin{tikzcd}
0\arrow{r} & \Z\arrow{r}{2\pi i \cdot} & \C \arrow{r}{\exp} & \C^{\times} \arrow{r} & 1.
\end{tikzcd}
\]
Consider the lattice $\Gamma = \{ m \cdot 2\pi i + k \cdot \log \lambda^{-1} \}$. Then $\exp(z + \Gamma) =
\{ e^z e^{- k \log \lambda} \} = \{ \lambda^{-k} w \}$, where we have defined $w = \exp z$. Thus, translations
by $\Gamma$ in $\C$ correspond, under $\exp$, to the action of $\Z$ on $\C^{\times}$. We conclude that
$X \cong \C/\Gamma$.
\end{proof}


\subsection*{Exercise 2.1.7 in \cite{Huy}}
\emph{Show that any Hopf surface contains elliptic curves.}
\begin{proof}
Hopf manifolds are defined as $X_n = \C^n - \{0\} / \Z$, where $k \in \Z$ acts by $(z_1, \dots, z_n) \mapsto
(\lambda^k z_1, \dots, \lambda^k z_n)$, with $0<|\lambda|<1$.

Start with a Hopf surface $X_2 = \C^2 - \{0\} / \Z$. Define:
\[	Y = \{(0, z_2) | z_2 \neq 0\} \subset \C^2 - \{0\}.	\]
Then $Y \cong \C - \{0\}$ and $Y$ is invariant under the action of $\Z$. Hence $Y/\Z$ is a subvariety of
$X_2$, and $Y / \Z \cong X_1$. From Exercise 2.1.5 in \cite{Huy}, whose proof is included below for reference,
we know that $X_1$ is an elliptic curve. Hence $Y/\Z$ is an elliptic curve contained in $X_2$.

We can attempt to count all elliptic curves contained in $X_2$. Notice that we can define:
\[	Y_{(a:b)} = \{(z_1, z_2) | (z_1:z_2) \in \PP^1 \text{ and } az_1 + bz_2 = 0 \text{ for some } (a:b) \in \PP^1\}
\subset \C^2 - \{0\}.	\]
$Y_{(a:b)}$ is again invariant under the action of $\Z$, and $Y_{(a:b)} \cong \C-\{0\}$. Thus we obtain an elliptic curve
$Y_{(a:b)}$ contained in $X_2$ for every $(a:b) \in \PP^1$.

We claim that these are actually all the irreducible algebraic curves that $X_2$ contains. Indeed, 
a subset $Y$ of $\C^2 - \{0\}$
defined by the vanishing of a polynomial $f(z_1, z_2)$ is invariant under the action of $\Z$ iff $f$ is
homogenous. Any homogenous polynomial in 2 variables factors into a product of linear factors, so:
\[	Y = V(f) = V\left(\prod_{i=1}^d (a_i z_1 + b_i z_2) \right)	= \bigcup_{i=1}^d Y_{(a_i:b_i)}. \]
Hence $Y$ is the union of $d$ of the elliptic curves defined in the previous paragraph.

In higher dimensions, we also obtain elliptic curves contained in $X_n$, defined by the vanishing of $n-1$
coordinates. (Or, more generally, the vanishing of a linear prime ideal of height $n-1$.)
\end{proof}



\subsection*{Exercise 2.1.8 in \cite{Huy}}
\emph{Describe the quotient of the torus $\C/\Z + \tau \Z$ by the involution $z \mapsto -z$ locally and
globally. Using the Weierstrass function $\wp$ again, one can show that the quotient is isomorphic to $\PP^1$.
What happens in higher dimensions?}
\begin{proof}
Let $\sigma$ denote the involution $z\mapsto -z$. Working in the fundamental cell of the flat torus, the
action of $\sigma$ has four fixed points given by $0, \frac{1}{2}, \frac{\tau}{2}, \frac{1+\tau}{2}$. In
a small neighborhood $U$ of any of these points $p$, $\sigma$ performs a rotation by $\pi$ centered at $p$. 
Therefore the quotient has one point for each pair of points in $U$ which are diametrally opposed with respect
to $p$. Away from the fixed points, each point has a neighborhood that is isomorphic and disjoint to its
image under $\sigma$.

We claim that $X = (\C / \Z + \tau \Z) / \langle \sigma \rangle$ is isomorphic to $\PP^1$. To see this, consider
the meromorphic function $\wp$ on $\C / \Z + \tau \Z$, which also gives a holomorphic map $\C / \Z + \tau \Z 
\to \PP^1$:
\[	\wp(z) = \frac{1}{z^2} + \sum_{\omega \in \Lambda^{\times}} \left[ \frac{1}{(z - \omega)^2} 
- \frac{1}{\omega^2}\right].	\]
We defined $\wp$ on $\C$, but it is doubly periodic, so it descends to a meromorphic map on the torus.
Moreover, $\wp$ is invariant under $\sigma$, so it descends to a holomorphic map $\bar \wp: X \to \PP^1$. $\bar \wp$ is 
not constant, so it is an open map. Since $X$ is compact, $\bar \wp(X)$ is a compact subset of the Hausdorff space
$\PP^1$, hence closed. Then $\bar \wp(X)$ is clopen in $\PP^1$, which shows that $\bar \wp$ is surjective. It
remains to show that $\bar \wp$ is injective, since a bijective holomorphic map must be an isomorphism. But injectivity
simply follows from the fact that $\wp :\C / \Z + \tau \Z \to \PP^1$ is a map of degree 2.
\end{proof}


\subsection*{Exercise 2.1.13 in \cite{Huy}}
\emph{Let $G = \langle \rho \rangle$, where $\rho$ is a primitive fifth root of unity. Let $\tilde G$ be the following
subgroup of $G^5$:}
\[	\tilde G = \left\{ (\xi_0, \dots, \xi_4) | \xi_i \in G, \prod_{i=0}^4 \xi_i = 1 \right\}.	\]
\emph{We let $\tilde G$ act on $\PP^4$ by $(z_0:\dots:z_4) \mapsto (\xi_0 z_0 : \dots : \xi_4 z_4)$. Describe the
subgroup $H$ that acts trivially. Show that the hypersurface}
\[	X = V \left( \sum_{i=0}^4 z_i^5 - 5t \prod_{i=0}^4 z_i \right) \subset \PP^4,	\]
\emph{with $t \in \C$, is invariant under $\tilde G$. Study the action of $\tilde G / H$ on $X$, in particular the
points with non-trivial stabilizer.}
\begin{proof}
To find the subgroup $H$ that acts trivially, consider first the case $\xi_0 \neq \xi_1$. Then, in general,
$(\xi_0 z_0 : \xi_1 z_1 : \dots) \neq (z_0 : z_1 : \dots)$. The same holds if
$\xi_i \neq \xi_j$ for other $i,j$. So $H = \{(\xi, \dots, \xi) | \xi \in G\}$. Note that $|G| = 5^5$, $\tilde G$
is a subgroup of index 5, so $|\tilde G| = 5^4$, and finally $H$ is a subgroup of order 5, so $|\tilde G /H| = 5^3$.

$X$ is invariant under $\tilde G$ because:
\begin{align*}
\sum_{i=0}^4 (\xi_i z_i)^5 - 5t \prod_{i=0}^4 (\xi_i z_i) &= \sum_{i=0}^4 \xi_i^5 z_i^5 - 5t \prod_{i=0}^4 \xi_i 
\prod_{i=0}^4 z_i \\
&= \sum_{i=0}^4 z_i^5 - 5t \prod_{i=0}^4  z_i.
\end{align*}
Let $x = (z_0:\dots:z_4) \in X$. We analyze 4 cases:
\begin{enumerate}
\item If none of the $z_i$ is zero, then $x$ is not fixed by any nontrivial element of $\tilde G/H$. 
We have $(\tilde G/H)_x = 1$ and $|\tilde G / H \cdot x| = 5^3$.
\item If one of the $z_i$ is zero, say $z_0 = 0$, then $(0:z_1:\dots:z_4) = (0:\xi_1z_1:\dots:\xi_4z_4)$
iff $\xi_1 = \dots = \xi_4$. Then the constraint $\prod_{i=0}^4 \xi_i = 1$ forces $\xi_0 = \xi_1$, so that
$(\xi_0, \dots, \xi_4) \in H$. Therefore $(\tilde G/H)_x = 1$ and $|\tilde G / H \cdot x| = 5^3$.
\item If two of the $z_i$ are zero, say $z_0 = z_1 = 0$, then $(0:0:z_2:z_3:z_4) = (0:0:\xi_2z_2:\xi_3z_3:\xi_4z_4)$
iff $\xi_2 = \xi_3 = \xi_4$. We can choose a representative for $(\xi_0, \dots, \xi_4)$ in $\tilde G/H$ such that
$\xi_2 = \xi_3 = \xi_4 = 1$. The remaining constraint is $\xi_0 \xi_1 = 1$. Therefore $(\tilde G/H)_x = \langle
\rho, \rho^{-1}, 1,1,1\rangle$, so $(\tilde G/H)_x = 5$ and $|\tilde G / H \cdot x| = 5^2$.
\item If three of the $z_i$ are zero, say $z_0 = z_1 = z_2 = 0$, then $x$ is stabilized iff $\xi_3 = \xi_4$. We
can choose a representative for $(\xi_0, \dots, \xi_4)$ in $\tilde G/H$ such that
$ \xi_3 = \xi_4 = 1$. The remaining constraint is $\xi_0 \xi_1 \xi_2 = 1$. Therefore $(\tilde G/H)_x = \langle
\rho, \rho', \rho^{-1}\rho'^{-1},1,1\rangle$, so $(\tilde G/H)_x = 5^2$ and $|\tilde G / H \cdot x| = 5$.
\end{enumerate}
This list is exhaustive. If four of the $z_i$ were zero, then the equation $\sum_{i=0}^4 z_i^5 - 
5t \prod_{i=0}^4 z_i = 0$ would force the fifth to be zero, so $x \not \in \PP^4$. We can conclude that the action
of $\tilde G / H$ partitions $X$ in three:
\begin{itemize}
\item an open subset where the action is free;
\item a codimension 2 subset (finite union of curves), given by the vanishing of 2 coordinates, where the stabilizer has order 5;
\item a codimension 3 subset (discrete), given by the vanishing of 3 coordinates, where the stabilizer
has order 25. In fact, for each pair of coordinates $z_i, z_j$ which can be nonzero, we obtain 5 points in this
subset, given by $z_j = \zeta z_i$, where $\zeta$ is a fifth root of $-1$. Since there are 10 ways to choose
$i,j$, we obtain $50$ points in this subset.
\end{itemize}
\end{proof}




\subsection*{Exercise 2.2.2 in \cite{Huy}}
\emph{Show that any short exact sequence of holomorphic vector bundles $0 \to L \to E \to F \to 0$ where
$L$ is a line bundle, induces short exact sequences of the form:}
\[
\begin{tikzcd}
0\arrow{r} & L\otimes \bigwedge^{k-1} F \arrow{r} & \bigwedge^{k} E\arrow{r} & \bigwedge^{k} F\arrow{r} & 0.
\end{tikzcd}
\]
\begin{proof}
Let $A : L \to E$ and $B:E \to F$ be the maps in the given short exact sequnce.
According to Example 2.2.4 ix) in \cite{Huy}, there exist trivializations of $L, E, F$ with transition maps
given by:
\[	
L \leftrightarrow \{\phi_{ij}\} \hspace{15mm} 
E\leftrightarrow 
\left\{\chi_{ij} =\left(\begin{array}{cc} \phi_{ij} & * \\ 0 & \psi_{ij} \end{array} \right) \right\} \hspace{15mm}
F \leftrightarrow \{\psi_{ij}\}. 	\]
Working over a patch $U_i$ in such a trivialization, define the following maps fiberwise, and omit the points
on the base in order to keep the notation as clean as possible.
\begin{align*}
\alpha: L \otimes \bigwedge^{k-1} F & \to \bigwedge^{k} E \\
(l\otimes f_1 \wedge \dots \wedge f_{k-1}) &\mapsto \big(A(l) \wedge \tilde f_1 \wedge \dots \wedge \tilde f_{k-1} \big) \\
\beta: \bigwedge^{k} E &\to \bigwedge^{k} F \\
(e_1 \wedge \dots e_k) &\mapsto \big(B(e_1) \wedge \dots \wedge B(e_k) \big).
\end{align*}
The maps extend uniquely by linearity to sums of simple tensors.
Here $\tilde f$ is any element of $B^{-1}(f)$. To see that $\alpha$ is well-defined, note that any other choice
of a preimage for $f$ differs by an element of $L$; $\tilde f' - \tilde f = A(l')$. The difference $A(l')$ gets
annihilated by the wedge product with $A(l)$, using the fact that the fiber of $L$ is one-dimensional.

Having defined $\alpha$ and $\beta$ fiberwise, we must check that that they commute with transition maps. Throughout
we use the fact that $A$ and $B$ do.
\begin{align*}
\phi_{ij} l \otimes \psi_{ij} f_1 \wedge \dots \wedge \psi_{ij} f_{k-1} &\mapsto A(\phi_{ij} l) \wedge
\widetilde{\psi_{ij} f_1} \wedge \dots \wedge \widetilde{\psi_{ij} f_{k-1}} \\
&= \phi_{ij} A(l) \wedge \psi_{ij}^{\wedge k-1}(\tilde f_1 \wedge \dots \wedge \tilde f_{k-1}) \\
&= \chi_{ij}^{\wedge k} A(l) \wedge \tilde f_1 \wedge \dots \wedge \tilde f_{k-1}.
\end{align*}
The argument for $\beta$ is analogous. Thus $\alpha$ and $\beta$ are morphisms of holomorphic vector bundles.
From the fiberwise construction it is immediate that $\alpha$ is injective and $\beta$ is surjective, using the
fact that $A$ and $B$ are. It remains to show that the sequence:
\[
\begin{tikzcd}
0\arrow{r} & L\otimes \bigwedge^{k-1} F \arrow{r}{\alpha} & \bigwedge^{k} E\arrow{r}{\beta} & \bigwedge^{k} 
F\arrow{r} & 0
\end{tikzcd}
\]
is exact in the middle. First, $\beta \circ \alpha = 0$ since $B(A(l)) = 0$ for all $l$. Consider now
$\beta(e_1 \wedge \dots \wedge e_k) = 0$. Since the fiber over $L$ is one-dimensional, and splitting 
$e_1 \wedge \dots \wedge e_k$ into a sum of terms if necessary, we can assume that all but one
of the $e_i$ lie in the orthogonal complement of $A(L)$. Say $e_1 = A(l) + e_1'$, where $e_1'$ and $e_2,
\dots, e_{k}$ are in the orthogonal complement of $A(L)$. Then:
\begin{align*}
0 = \beta(e_1 \wedge \dots \wedge e_k) &= \beta(A(l) \wedge \dots \wedge e_k) + \beta(e_1' \wedge \dots \wedge e_k) \\
&= \beta(e_1' \wedge \dots \wedge e_k) \\
&= e_1' \wedge \dots \wedge e_k.
\end{align*}
Hence we must have $e_1' \wedge \dots \wedge e_k = 0$. Therefore $e_1 \wedge \dots \wedge e_k = A(l) \wedge \dots 
\wedge e_k \in \im(\alpha)$.
\end{proof}


\subsection*{Exercise 2.2.5 in \cite{Huy}}
\emph{Let $L$ be a holomorphic line bundle on a compact complex manifold $X$. Show that $L$ is trivial if and
only if $L$ and its dual $L^*$ admit non-trivial global sections. (Hint: Use the non-trivial sections to construct
a non-trivial section of $\mathcal{O} \cong L \otimes L^*$.)}

\begin{proof}
If $L \cong X \times \C$ is trivial, then $s : X \to L$ given by $s(x) = (x,1)$ is a nontrivial global section
of $L$. $L^*$ is also trivial, since its transition maps are $\id ^{-1} = \id$. Hence we can take $s^* : X \to L^*$
given by $s^*(x) = (x,1)$.

Conversely, assume that $L$ and $L^*$ have nontrivial global sections $s$ and $s^*$. Define $f \in \mathcal{O}_X(X)$
as follows. Choose an open cover $\{U_i\}$ of $X$ which trivializes $L$ and $L^*$, and let $p_i, p_i^*$ denote
projections to the fiber direction. Then let $f_i(x) = \big(p_i^* \circ s^*(x)\big) \big(p_i \circ s(x)\big)$.
The $f_i$ so defined on $U_i$ glue to a holomorphic function on $X$, since the transition maps of $L$ and $L^*$
cancel out to give $f_i = f_j$ on $U_{ij}$. Hence $f \in \mathcal{O}_X(X)$, so $f$ is constant, using the compactness
of $X$. The value of
the constant cannot be zero, since that would mean that either $s$ or $s^*$ is a trivial section. Hence
$f(x) \neq 0$ for all $x$, which, together with the fact that $p_i^* \circ s^*(x)$ is a linear functional on
the fiber $L_x$, shows that $p_i \circ s(x) \neq 0$ for all $x$. This allows us to construct an isomorphism 
$L \cong X \times \C$, by mapping $s(x)$ to $(x,1)$.
\end{proof}


\subsection*{Exercise 2.2.8 in \cite{Huy}}
\emph{Show that any non-trivial homogenous polynomial $0\neq s \in \C[z_0, \dots, z_n]_k$ of degree $k$ can be
considered as a non-trivial section of $\mathcal{O}(k)$ on $\PP^n$. (In fact, all sections are of this form, cf.
Proposition 2.4.1.)}

\begin{proof}
By definition, a section of $\mathcal{O}(1)$ gives a linear functional on each fiber of $\mathcal{O}(-1)
= \{(\ell, z) \in \PP^n \times \C^{n+1} | z \in \ell \}$. Let $p_i : \C^{n+1} \to \C$ denote the projection
onto the $i^{\text{th}}$ factor. Then $z_i \in \C[z_0, \dots, z_n]_1$ is a linear functional on each fiber,
by regarding it as $(\ell, z) \mapsto p_i(z)$. Hence $z_i$ is a section of $\mathcal{O}(1)$. Moreover, $z_i$
is a holomorphic section, a property which can be checked locally by evaluating $z_i$ on a holomorphic local 
section of $\mathcal{O}(-1)$. We obtain $z_i \in H^0\big(X, \mathcal{O}(1)\big)$ for all $i$. Then
$\C[z_0, \dots, z_n]_1 \subset H^0\big(X, \mathcal{O}(1)\big)$, because the latter is a vector space over $\C$.

Given $s_1, \dots, s_k \in H^0\big(X, \mathcal{O}(1)\big)$, we obtain a section $s_1 \otimes \ldots \otimes s_k
\in H^0\big(X, \mathcal{O}(k)\big)$, defined fiberwise by:
\begin{equation}
\label{eq:sectiontautology}
	s_1 \otimes \ldots \otimes s_k (z_1, \dots, z_k) = s_1(z_1) \cdot \ldots \cdot s_k(z_k).
\end{equation}
Taking $s_1, \dots, s_k \in \C[z_0, \dots, z_n]_1$, we have $s_1 \otimes \ldots \otimes s_k \in \C[z_0, \dots,
z_n]_k$, and the construction in \ref{eq:sectiontautology} gives $s_1 \otimes \ldots \otimes s_k \in 
H^0\big(X, \mathcal{O}(k)\big)$. Since every element of $\C[z_0, \dots,
z_n]_k$ is a sum of elements of the form $s_1 \otimes \ldots \otimes s_k$, and $H^0\big(X, \mathcal{O}(k)\big)$
is a vector space, we have proved that $\C[z_0, \dots,z_n]_k \subset H^0\big(X, \mathcal{O}(k)\big)$.
\end{proof}


\subsection*{Exercise 2.2.10 in \cite{Huy}}
\emph{Let $\{(U_i, \phi_i)\}$ be an atlas of the complex manifold $X$. Use the cocycle description of the
holomorphic tangent bundle $\mathcal{T}_X$ to show that for $x\in U_i \subset X$ the fibre $\mathcal{T}_X(x)$
can be identified with $T_{\phi_i(x)}\phi_i(U_i) \cong T^{1,0}_{\phi_i(x)}\phi_i(U_i)$. In particular, the
vectors $\p/\p z_i$ can be viewed as a basis of $\mathcal{T}_X(x)$.}

\begin{proof}
We use the fact that the tangent bundle of $\phi_i(U_i) \subset \C^n$ is trivial. Specifically, $\{\p_{z_i}\}$
give a global frame for the tangent bundle of $\phi_i(U_i)$, which we can use to give an explicit local
trivialization of $\mathcal{T}_X$:
\begin{align}
\label{eq:tangenttautology}
\begin{split}
U_i \times \C^n &\to \mathcal{T}_X|_{U_i}  \\
\big(x, (a_1, \dots, a_n)\big) &\mapsto \big(\phi_i^{-1}(x), d{\phi_i^{-1}}_x (a_1 \p_{z_1} + \dots + a_n 
\p_{z_n})\big).
\end{split}
\end{align}
$d{\phi_i^{-1}}_x$ is an isomorphism because $\phi_i$ is a coordinate chart for $X$. Note that the transition
maps associated to this local trivialization are indeed $d{\phi_j}_{\phi_i^{-1}(x)} \circ d{\phi_i^{-1}}_x 
= d{\phi_{ij}}_x$, so \ref{eq:tangenttautology} gives the correct cocycle description of $\mathcal{T}_X$.
Moreover, \ref{eq:tangenttautology} describes each fiber of $\mathcal{T}_X$ as the image of $T_{\phi_i(x)}
\phi_i(U_i)$ under the isomorphism $d{\phi_i^{-1}}_x$.
\end{proof}

\subsection*{Exercise 2.2.12 in \cite{Huy}}
\emph{Let $Y \subset X$ be a submanifold locally in $U \subset X$ defined by holomorphic functions
$f_1, \dots, f_{n-k}$ (i.e. 0 is a regular value of $(f_1, \dots, f_{n-k}) : U \to \C^{n-k}$ and $Y$ is the
preimage of it). Show that $f_1, \dots, f_{n-k}$ naturally induce a basis of $\mathcal{N}^*_{Y|X}(x)$ for
any $x \in Y \cap U$. (Use the map $\p_{z_i} \to \p_{z_i} f_j$ for $i = k+1, \dots, n$.)}

\emph{Go on and prove the existence of a natural isomorphism $\mathcal{N}^*_{Y|X} \cong 
\mathcal{I}_Y/\mathcal{I}_Y^2$.}

\begin{proof}
Let $f$ be the local map with components $f_1, \dots, f_{n-k}$. Consider $df_x : \mathcal{T}_X|_Y(x)
\to \C^{n-k}$. Its kernel is $\mathcal{T}_Y(x)$, because $Y$ is locally a level set of $f$. Thus
$df_x$ descends to an isomorphism:
\begin{align*}
\mathcal{T}_X|_Y(x) / \mathcal{T}_Y(x) \cong \mathcal{N}_{Y|X}(x) &\to \C^{n-k} \\
\p_{z_i} &\mapsto \frac{\p f_j}{\p z_i}, \;\; i= k+1, \dots, n.
\end{align*}
In this way the components of $df_x$ give a basis of $\mathcal{N}^*_{Y|X}(x)$. Note also that this identification
is coordinate independent.

We obtain a map $d_x : \mathcal{I}_{Y,x} \to \mathcal{N}^*_{Y|X, x}(U)$, which sends $f$ to the differential
$df_x$ given by $df_x(\p_{z_i}) = \p_{z_i}f(x)$. The kernel is $\{f \in \mathcal{I}_{Y,x}| \p_{z_i}f (x)= 0 
\text{ for } 
i = k+1, \dots, n\}$. Using the Taylor expansion of $f$, this is equivalent to $f \in \mathcal{I}_{Y,x}^2$.
Therefore $d_x$ descends to an isomorphism $\bar d_x : (\mathcal{I}_Y/\mathcal{I}^2_Y)_x \overset{\sim}{\to}
{N}^*_{Y|X,x}$. We now have a continuous isomorphism of the stalks of $\mathcal{I}_Y/\mathcal{I}^2_Y$ and 
${N}^*_{Y|X}$. This gives an isomorphism of sheaves. \todo{read more about this and give a reference}
\end{proof}


\subsection*{Exercise 2.2.13 in \cite{Huy}}
\emph{Show that the holomorphic tangent bundle of a complex torus $X= \C^n/\Gamma$ is trivial, i.e. isomorphic
to the trivial vector bundle $\mathcal{O}^{\oplus n}$. Compute $\kod(X)$.}

\begin{proof}
Consider the vector fields $\{\p/\p z_i\}$ on $\C^n$. They are translation invariant, so they descend to
vector fields $\{V_i\}$ on $X$. Moreover, $\{\p/\p z_i\}$ form a basis for $T_z\C^n$ at every $z$, so their
images $\{V_i\}$ under the quotient map (a local isomorphism) also do. Therefore the $\{V_i\}$ provide
a global frame for $\mathcal{T}_X$, which must be trivial.

An analogous argument shows that the cotangent bundle $\mathcal{T}^*_X$ is trivial, with a global frame
given by the images $\{\omega_i\}$ of $\{dz_i\}$. Hence the following is an isomorphism of $\mathcal{O}_X$-modules:
\begin{align*}
\mathcal{O}_X^{\oplus n}(U) &\to \mathcal{T}^*_X(U) \\
(f_1, \dots, f_n) &\mapsto \sum_{i=1}^n f_i \omega_i|_U.
\end{align*}
$K_X = \det \mathcal{T}^*_X$ is also trivial, because its transition functions are the determinant of the
identity map. Thus $K_X \cong \mathcal{O}_X$, and in particular $H^0(X, K_X) \cong \C$. Denote by $\omega$ a
choice of generator of $H^0(X, K_X)$ over $\C$. We have $K_X^{\otimes j} \cong \mathcal{O}_X^{\otimes j} \cong
\mathcal{O}_X$, and $H^0(X, K_X^{\otimes n}) \cong \C$, generated over $\C$ by $\omega^{\otimes n}$. This means
that $R(X) = \C[\omega]$, and $Q(X) = \C(\omega)$. In particular, $\kod(X) = 1 - 1 = 0$.
\end{proof}


\subsection*{Problem 2.3.2 in \cite{Huy}}
For want of a better idea, we follow the argument given in Proposition 2.4.7 in \cite{Huy}, which uses the cocycle
description of both line bundles.

To describe the cocycle of $\mathcal{N}_{Y|X}$, we choose local coordinates $\phi_i$ around all $p \in Y$ such that
$Y = Z(\phi^n_i)$ locally, where $\phi^n_i$ is the last component. Then the transition functions satisfy
$\phi_{ij}^n(z_1, \dots, z_{n-1}, 0) = 0$, so $\phi^n_{ij}(z) = z_n \cdot h(z_1, \dots, z_n)$, for some holomorphic
$h$. Therefore all partial derivatives of $\phi^n_{ij}$ except for the $n^{\text{th}}$ one evaluate to zero
on $Y$, where we have $z_n = 0$. The Jacobian matrix takes the form:
\[
J(\phi_{ij})|_Y = \left( \begin{array} {c|c}
J(\phi_{ij}|Y) & * \\ \hline
0 & \p_{z_n} \phi^n_{ij}
\end{array} \right) .
\]
Since the top left block is the cocycle of $\mathcal{T}_Y$, the bottom right block is the cocycle of $\mathcal{N}
_{Y|X}$.

Since $Y$ is defined by the vanishing of a some $s \in \Gamma\big(\mathcal{O}(Y)\big)$, the cocycle of the latter
is given by $s_i/s_j$, where the $s_i$ are local coordinate descriptions of the section $s$. In particular,
$s_i = \phi_i^n$. Hence for $z \in Y$ we have:
\begin{align*}
\frac{s_i}{s_j}(z) &= \frac{(\phi_{ij} \circ \phi_j)^n}{\phi_j^n}(x) = \frac{\phi_{ij}^n}{z_n} \phi_j(x) \\
&= h(z_1, \dots, z_{n-1}, 0) = \p_{z_n} \phi_{ij}^n(z_1, \dots, z_{n-1}, 0).
\end{align*}
$\mathcal{O}(Y)|_Y$ and $\mathcal{N}_{Y|X}$ are given by the same cocycle, so they are isomorphic.



\subsection*{Problem 2.3.3 in \cite{Huy}}
Let $X_i = V(f_i)$, for $1\leq i \leq k$, and let $X_{i_1 \dots i_l} = X_{i_1} \cap \dots \cap X_{i_l}$. We
claim that $\mathcal{N}_{X_{1\dots k}|\PP^N} \cong \bigoplus_{i=1}^k \mathcal{N}_{X_i|\PP^N}|_{X_{1\dots k}}$.
Indeed, since $X_{1\dots k}$ is a complete intersection, the derivatives of $f_1, \dots, f_k$ are linearly
independent at each point. On some open set $U$ around each $p \in X_1 \cup \dots \cup X_k$, we can find functions 
$g_{k+1}, \dots, g_N$ such that the derivatives of all $N$ functions are linearly independent in $U$.
\footnote{$f_1, \dots, f_k$ are not strictly speaking functions on $\PP^N$, but on $\C^{N+1} - \{0\}$. However,
we may regard them locally as functions on $\PP^N$ by choosing appropriate affine coordinates. The derivatives
remain linearly independent in these coordinates.} Using the inverse function theorem, the map $\phi : U \to \C^N$
which has components $(f_1, \dots, f_k, g_{k+1}, \dots, g_N)$ is a local diffeomorphism, and we adopt it as a
coordinate chart around $p$. Then $X_i = V(\phi_i)$ locally.

Generalizing the argument made in Problem 2.3.2, the $i^{\text{th}}$ component of the transition functions $\phi_{lm}$
satisfies $\phi_{lm}^i(z_1, \dots, z_{i-1}, z_i = 0, z_{i+1}, \dots, z_{N}) = 0$ for $1\leq i \leq k$, which gives
$\phi_{lm}^i(z_1, \dots, z_{N}) = z_i \cdot h^i_{lm}(z_1, \dots, z_{N})$. Hence all partial derivatives
of $\phi_{lm}^i$ except for the $i^{\text{th}}$ one are zero when restricted to $X_{1\dots k}$. Therefore:
\[	
J(\phi_{lm})|_{X_{1\dots k}} = \left( \begin{array} {cccc|c} 
\p_{z_1}\phi_{lm}^1 & 0 & \dots & 0 & 0 \\
0 & \p_{z_2}\phi_{lm}^2 & \dots & 0 & 0 \\
\vdots & \vdots & \ddots & 0 & 0 \\
0 & 0 & \dots & \p_{z_k}\phi_{lm}^k & 0 \\ \hline
* & * & * & * & J(\phi_{lm}|_{X_{1\dots k}})
\end{array} \right)	.
\]
The lower right block is the cocycle of $\mathcal{T}_{X_{1\dots k}}$ so, by definition, the upper left block 
is the cocycle of $\mathcal{N}_{X_{1\dots k}|\PP^N}$. We can read off the decomposition
as $\oplus_{i=1}^k \mathcal{N}_{X_{i}|\PP^N}|_{X_{1\dots k}}$. Using the result of Exercise 2.3.2 in \cite{Huy}
, each direct
summand is isomorphic to $\mathcal{O}_{\PP^N}(d_i)|_{X_{1\dots k}}$.




\subsection*{Problem 2.3.8 in \cite{Huy}}
Let $s_1 \in \Gamma\big(\mathcal{O}_{\PP^2}(1)\big)$ and $s_d \in \Gamma\big(\mathcal{O}_{\PP^2}(d)\big)$; moreover,
let $H = V(s_1)$, $C = V(s_d)$ and $B = H \cap C = V(s_1, s_d)$. We need to show that $\mathcal{O}_{\PP^2}(1)|_C$
has degree $d$, which is to say that any divisor of $C$ associated to a nonzero section of $\mathcal{O}_{\PP^2}(1)|_C$ 
has degree $d$. Clearly $B$ is such a divisor, so we are interested in $\deg B$. However, $B$ is also a 
divisor on $H$, and
$\deg B$ is the same on both $C$ and $H$, because $B$ is a formal sum of points. Then it suffices to consider
$B$ as a divisor of $H$.

Using the fact that automorphisms of $\PP^2$ are transitive on hyperplanes\footnote{$\Aut(\PP^n) = \PGL(n+1)$ 
is transitive
on collections of $n$ points, and $n$ points determine a hyperplane in $\PP^n$.}, we can take $s_1 = z_2$, and $H$
is the copy of $\PP^1$ inside $\PP^2$ given by $z_2 = 0$. Then $\mathcal{O}_{\PP^2}(d)|_H \cong \mathcal{O}_{\PP^1}
(d)$, as can be seen by inspecting global sections: those that contain $z_2$ restrict to $0$, and the ones left are
$\langle z_0^d, z_0^{d-1}z_1, \dots, z_0 z_1^{d-1}, z_1^d \rangle_{\C}$. $B \in \Div(H)$ is given by the vanishing
of one of these sections, hence $\deg B = d$. For example, $V(z_0^d) = d [0:1]$, and $V(z_0^{d-1}z_1) = (d-1)[0:1]
+ [1:0]$. We conclude that $\mathcal{O}_{\PP^2}(1)|_C$ has degree $d$.

Now let $D = V(s_e)$ where $s_e \in \Gamma\big(\mathcal{O}_{\PP^2}(e)\big)$. Since $\mathcal{O}(B + B') =
\mathcal{O}(B) \otimes \mathcal{O}(B')$, the result of the previous paragraph implies that $\mathcal{O}_{\PP^2}(e)
|_C$ has degree $de$. The intersection divisor defined below is associated to $\mathcal{O}_{\PP^2}(e)|_C$, so it has
degree $de$:
\begin{equation}
\label{eq:bezadhoc}
	de = \deg \sum_{p \in C \cap D} \ord_{s_e} (p \in C)\cdot	p.
\end{equation}
By $\ord_{s_e} (p \in C)$ we mean the order of vanishing of $s_e$ at $p$, where $s_e$ is is seen as a section
of $\mathcal{O}_{\PP^2}(e)|_C$. Since $p$ is an intersection point, this order is $\geq 1$. We want to give a
more invariant interpretation of $\ord_{s_e} (p \in C)$, and for this we use stalks. The stalk at $p$ is a local
concept, so it suffices to analyze the stalk $\mathcal{O}_{\C^2, p}$ of the sheaf of holomorphic functions on an
affine neighborhood of $p$. Using the Weierstrass division theorem (1.1.17 in \cite{Huy}), we can write any
$f \in \mathcal{O}_{\C^2, p}$ as $f = s_d g + r$, where the vanishing order of $r$ along $V(s_d)$ is lower than
that of $g$. Moreover, in the quotient $\mathcal{O}_{\C^2, p}/(s_d)$, we apply the same theorem to obtain
$r = s_e h + r'$, where $\ord_{r'} (p \in C) < \ord_{s_e} (p \in C)$. Therefore the succesive quotient is:
\[	\mathcal{O}_{\C^2, p}/(s_d,s_e) \cong \{[r'] | \ord_{r'} (p \in C) < \ord_{s_e} (p \in C).	\]
The equivalence classes are parametrized by the order of vanishing at $p$ along $C$, so they form an
$\ord_{s_e} (p \in C)$ vector space over $\C$. With this we can rewrite \ref{eq:bezadhoc} as:
\[	de = \deg \sum_{p \in C \cap D} \dim_{\C} \mathcal{O}_{\C^2, p}/(s_d,s_e) \cdot	p.	\]
Throughout we have been implicitly using the fact that $C$ and $D$ are distinct smooth curves, so $s_d$ and $s_e$
are distinct irreducible homogenous polynomials. This ensures that the Krull dimension of $\mathcal{O}_{\C^2, p}$
drops by 1 with each quotient, so $\mathcal{O}_{\C^2, p}/(s_d,s_e)$ is indeed a finite dimensional vector space
over $\C$.




\subsection*{Problem 2.3.10 in \cite{Huy}}
\emph{Let $C = \phi_{\mathcal{O}(2)}(\PP^1) \subset \PP^2$ and consider the restriction of the linear system in
Exercise 2.3.7 to $C$. Study the induced map $C - \{x\} \to \PP^1$. (There are two cases to be considered: $x \in C$
and $x \not \in C$.)}
\begin{proof}
To avoid confusion, we denote by $z_i$ the homogenous coordinates on $\PP^1$ and by $x_i$ the homogenous coordinates
on $\PP^2$.

Recall that $H^0(\PP^1, \mathcal{O}(2)) \cong \C[z_0, z_1]_2 = \langle z_0^2, z_0z_1, z_1^2 \rangle_{\C}$. There
is no $(z_0:z_1) \in \PP^1$ such that all 3 basis elements vanish at the same time, so the base locus is empty.
Therefore $\phi_{\mathcal{O}(2)} : \PP^1 \to C$ is an isomorphism, which we recognize as the Veronese embedding
of $\PP^1$ in $\PP^2$:
\begin{align*}
\phi_{\mathcal{O}(2)} : \PP^1 &\to \PP^2 \\
(z_0:z_1) &\mapsto (z_0^2:z_0z_1:z_1^2).
\end{align*}
The map $\phi$ of Exercise 2.3.7 restricts differently to $C$ based on whether $x \in C$ or not. We claim that
it suffices to analyze one representative for each of the two cases. When $x \in C$, let $(z_0:z_1) \in \PP^1$ be
the unique point such that $\phi_{\mathcal{O}(2)}(z_0:z_1) = x$. Then there exists an automorphism of $\PP^1$ that
takes $(z_0:z_1) \mapsto (0:1)$, so precomposing with this automorphism gives $x = \phi_{\mathcal{O}(2)}(0:1) =
(0:0:1)$. When $x \not \in C$, I don't see a way of using automorphisms of $\PP^2$ to bring $x$ to a predetermined
point, say $(0:1:0)$, while at the same time preserving $C$. However, I'm confident that there exists some
argument by symmetry that justifies only analyzing $(0:1:0)$. (See last paragraph for an attempt.)

For $x = (0:0:1)$, the linear system is:
\[	L = \langle x_0, x_1 \rangle_{\C}.	\]
The corresponding map $\phi : \PP^2 - \{(0:0:1)\} : \PP^1$ takes $(x_0 : x_1 : x_2) \mapsto (x_0:x_1)$. In particular,
the restriction to $C - \{(0:0:1)\}$ is $(z_0^2 : z_0z_1 : z_1^2) \mapsto (z_0^2:z_0z_1) = (z_0:z_1)$. We were able
to divide by $z_0$ because $z_0 \neq 0$ on $C - \{(0:0:1)\}$. Hence the composition $\phi|_{C - \{(0:0:1)\}} \circ
\phi_{\mathcal{O}(2)}$ restricts to the identity map on the affine open $\PP^1 - \{(0:1)\}$.

For $x = (0:1:0)$ the linear system is:
\[	L = \langle x_0, x_2 \rangle_{\C}.	\]
The corresponding map $\phi : \PP^2 - \{(0:1:0)\} : \PP^1$ takes $(x_0 : x_1 : x_2) \mapsto (x_0:x_2)$. In particular,
the restriction to $C$ is $(z_0^2 : z_0z_1 : z_1^2) \mapsto (z_0^2:z_1^2)$. Hence the composition
$\phi|_{C} \circ \phi_{\mathcal{O}(2)}$ is the following endomorphism of $\PP^1$:
\[	(z_0:z_1) \mapsto (z_0^2 : z_1^2).	\]
The map is surjective, but is not injective. In fact, it is a double cover of $\PP^1$ by itself, ramified at
$(0:1)$ and $(1:0)$.

We claim that the topological properties of $\phi|_{C} \circ \phi_{\mathcal{O}(2)}$ remain the same if we consider
projection from some other $x \not \in C$ instead. For example, take an automorphism of $\PP^2$ that brings
$x$ to $(0:1:0)$, and sends $C$ to some other quadratic curve $Q \subset \PP^2$, not passing through $(0:1:0)$. 
For fixed $a,b$, let $L$ be the line $(a:x_1:b)$; projection from $(0:1:0)$ sends all $p \in Q \cap L$ to
$(a:b)$. The number of intersections $p$, counting with multiplicity, is 2, using B\'ezout's theorem. This
confirms that we obtain a double cover of $\PP^1$ by itself, but it's harder to see how many ramification points
there are.
\end{proof}


\subsection*{Exercise 2.4.1 in \cite{Huy}}
\emph{Show that the canonical bundle $K_X$ of a complete intersection $X = Z(f_1) \cap \dots \cap Z(f_k)
\subset \PP^N$ is isomorphic to $\mathcal{O}(\sum \deg(f_i) - n - 1)|_X$. What can you deduce from this
about the Kodaira dimension of $X$?}

\emph{Extra question added by Ron Donagi: classify Calabi-Yau complete intersections in $\PP^N$, up to 
discrete invariants.}

\begin{proof}
We saw in Exercise 2.3.3 in \cite{Huy} that a complete intersection in $\PP^N$ satisfies:
$\mathcal{N}_{X_{1\dots k}|\PP^N} \cong \bigoplus_{i=1}^k \mathcal{O}_{\PP^N}(d_i)|_{X_{1\dots k}}$,
so we have:
\begin{align*}
\det \mathcal{N}_{X_{1\dots k}|\PP^N} &\cong \left[ \bigoplus_{j_1 + \dots + j_k = k}
 \mathcal{O}_{\PP^N}(d_1)^{\wedge j_1} \otimes \dots
\otimes \mathcal{O}_{\PP^n}(d_k)^{\wedge j_k} \right]_{X_{1\dots k}} \\
&= \left[ \bigotimes_{i=1}^k \mathcal{O}_{\PP^N}(d_i) \right]_{X_{1\dots k}} 
= \left[ \mathcal{O}_{\PP^N}\left(\sum_{i=1}^k d_i\right) \right]_{X_{1\dots k}}.
\end{align*}
The direct summands with $j_i > 1$ have vanished because the fibers of a line bundle are one dimensional, so cannot
sustain any higher wedge powers. Using the adjunction formula and the fact that $K_{\PP^N} \cong \mathcal{O}(-N-1)$,
we obtain that:
\[	K_{X_{1\dots k}} \cong \mathcal{O}_{\PP^N}(\sum_{i=1}^k d_i - N - 1) |_{X_{1\dots k}}.	\]

For the Kodaira dimension of $K_{X_{1\dots k}}$ we distinguish 3 cases.
\begin{itemize}
\item If $D := \sum_{i=1}^k d_i - N - 1 >0$, $K_{X_{1\dots k}}$ is the restriction to $X_{1\dots k}$ of
a bundle with nontrivial global sections; the pullback of at least one of these sections is nontrivial.
(See next paragraph for an argument.) Hence $\Kod(X_{1\dots k}) > 0$.
\item If $D := \sum_{i=1}^k d_i - N - 1 =0$, then $K_{X_{1\dots k}}$ is trivial, so $R(X_{1\dots k}) \cong
\C[x]$ and $\Kod(X_{1\dots k}) = 0$. (This agrees with the vague correspondence between Kodaira dimension
and constant curvature metrics. $\Kod(X) = 0$ loosely corresponds to manifolds which admit flat metrics,
and the Calabi conjecture gives the existence of a Ricci flat metric on $X$ when $K_X$ is trivial.)
\item If $D  < 0$, then $K_{X_{1\dots k}}$ is the dual of a nontrivial bundle which admits global sections.
Therefore $K_{X_{1\dots k}}$ itself does not admit global sections. We have
$R(X_{1\dots k})\cong \C$, so $\Kod(X_{1\dots k}) = - \infty$.
\end{itemize}

In order for $X_{1\dots k}$ to be a Calabi-Yau, we impose the condition $K_{X_{1\dots k}} \cong \mathcal{O}
_{X_{1\dots k}}$. This is fullfiled if $\sum_{i=1}^k d_i = N+1$, and we claim that the converse is also true, i.e.
that a nontrivial line bundle $\mathcal{O}_{\PP^N}(m)$ on $\PP^N$ restricts to a nontrivial line bundle on the 
complete intersection. It suffices to check for $m>0$, since $m<0$ corresponds to the dual bundle. We have
$\dim \Gamma\big(\mathcal{O}_{\PP^N}(m)\big) = \binom{N+m}{m} \geq N+1$. We choose $N+1$ linearly independent
global sections, which restrict to global sections on $X_{1\dots k}$. Some of these may become constant upon restriction,
but at least one does not: otherwise, $N+1$ linearly independent conditions give $X = \emptyset$.

We conclude that being a Calabi-Yau complete intersection is equivalent to 
\begin{equation}
\label{eq:CY}
\sum_{i=1}^k d_i = N+1.
\end{equation}
 This lets us
classify Calabi-Yau complete intersections in $\PP^N$ of any given dimension, as exemplified below.
\begin{enumerate}
\item Curves, i.e. $k = N-1$. The possibilities are:
\begin{itemize}
\item $d_1 = 3$, $d_2 = \dots = d_{N-1} = 1$. Taking intersections with hyperplanes is equivalent to having an
embedding in a lower dimensional projective space, so this gives a cubic (elliptic) curve in $\PP^3$.
\item $d_1 = d_2 = 2$, $d_3 = \dots = d_{N-1} = 1$. This is the intersection of two quadrics in $\PP^4$.
\end{itemize}
\item Surfaces, i.e. $k = N-2$. The possibilities are:
\begin{itemize}
\item $d_1 = 4$, $d_2 = \dots = d_{N-2} = 1$. This is a quartic in $\PP^4$.
\item $d_1 = 3$, $d_2 = 2$, $d_3 = \dots = d_{N-2} = 1$. This is the intersection of a cubic and a quadric
in $\PP^5$.
\item $d_1 = d_2 = d_3 = 2$, $d_4 = \dots = d_{N-2} = 1$. This is the intersection of three quadrics in
in $\PP^6$.
\end{itemize}
\item Threefolds, i.e. $k = N-3$. The possibilities are:
\begin{itemize}
\item $d_1 = 5$, $d_2 = \dots = d_{N-3} = 1$. This is a quintic in $\PP^5$.
\item $d_1 = 4$, $d_2 = 2$, $d_3 = \dots = d_{N-3} = 1$. This is the intersection of a quartic and a quadric in
$\PP^6$.
\item $d_1 =d_2 = 3$, $d_3 = \dots = d_{N-3} = 1$. This is the intersection of two cubics in $\PP^6$.
\item $d_1 =3$, $d_2 = d_3 = 2$, $d_4 = \dots = d_{N-3} = 1$. This is the intersection of a cubic and two quadrics
in $\PP^7$.
\end{itemize}
\item CY hypersurfaces in $\PP^N$ must have degree $N+1$.
\end{enumerate}
This is only a classification up to discrete invariants; each class given above actually corresponds to a moduli
space of Calabi-Yaus.
\end{proof}


\section{Huybrechts, Chapter 3}
\subsection*{Exercise 3.1.5 in \cite{Huy}}
\emph{Let $\C^n \to \C^{n+1}$ be the standard inclusion $(z_0,\dots, z_{n-1}) \mapsto (z_0,\dots,z_{n-1}, 0)$
and consider the induced inclusion $\PP^{n-1} \subset \PP^n$. Show that restricting the Fubini-
Study K\"{a}hler form $\omega_{FS}(\PP^n)$ yields the Fubini-Study K\"{a}hler form on $\PP^{n - 1}$.}
\vspace{3mm}

Let $j: \PP^{n-1} \to \PP^n$ be the inclusion, so that $\PP^{n-1}$ is the hyperplane $z_n = 0$. We look at $j^*\omega_{FS,i}
(\PP^n)$
on all affine charts $U_i$. Note that $U_n \cap j(\PP^{n-1}) = \emptyset$, so we only need to check 
on $U_0, \dots, U_{n-1}$. In these cases we have:
\[	j^* \omega_{FS,i}(\PP^n) = \frac{i}{2\pi}	j^* \p \bar \p \log \left( \sum_{k=0}^n \left| \frac{z_k}{z_i} \right| \right)
= \p \bar \p \log \left( \sum_{k=0}^{n-1} \left| \frac{z_k}{z_i} \right| \right) = \omega_{FS,i}(\PP^{n-1}). \]
These glue to give $\omega_{FS}(\PP^{n-1})$.


\subsection*{Exercise 3.1.7 in \cite{Huy}}
\emph{Show that $L$, $d$, and $d^*$ acting on $\mathcal{A}^*(X)$ of a K\"{a}hler manifold $X$ determine
the complex structure of $X$.}
\vspace{3mm}

This and some of the following problems are easy and short once you figure out which K\"{a}hler relations to use.
In this case we have:
\[	d^* = -* (\p + \bar \p) * = \p^* + \bar \p^*.	\]
Therefore:
\[	[d^*, L] = [\p^* + \bar \p^*, L] = i\p - i \bar \p.	\]
Then we can recover:
\[	\p = \frac{id + [d^*,L]}{2i} \hspace{1cm} \bar \p = \frac{id - [d^*,L]}{2i} .	\]
Knowledge of $\p$ and $\bar \p$ implies ability to tell which elements of $\mathcal{A}^*(X)$ are holomorphic, and this
determines the complex structure of $X$.



\subsection*{Exercise 3.1.8 in \cite{Huy}}
\emph{Show that on a compact K\"{a}hler manifold $X$ of dimension $n$ the integral
$\int_X \omega^n$ is $n! \text{vol}(X)$ (cf. Exercise 1.2.9). Conclude from this that there exists an
injective ring homomorphism $k[x]/x^{n+1} \to H^*(X,\R)$. In particular, $b_2(X) \geq 1$.
Deduce from this that $S^2$ is the only sphere that admits a K\"{a}hler structure.}
\vspace{3mm}

Following the approach of \cite{Voi1} in Lemma 3.8, we show that $\omega^n = n! \dVol$ locally. Given that $X$
is orientable, $\dVol$ for $X$ is then obtained from glueing the local forms, so it is also equal to $\omega^n$.

So let $e_1, \dots, e_n$ be a basis over $\C$ for the tangent space $T_{\C,0}$ such that $h(e_i, e_j) = \delta_{ij}$,
where $h$ is the hermitian form of the K\"{a}hler structure. Then $e_1, Ie_1, \dots, e_n, Ie_n$ is an oriented
basis over $\R$ for $T_{\C,0}$, which is orthonormal for $h$. Take $dx_1, dy_1, \dots, dx_n, dy_n$ to be the dual basis,
and let $dz_j = dx_j + idy_j$. Then we have:
\begin{align*}
\omega &= \frac{i}{2} \sum_{j=1}^n dz_j \wedge d\bar z_j = \sum_{j=1}^n dx_j \wedge dy_j. \\
\omega^n &= n! \prod_{j=1}^n dx_j \wedge dy_j = n! \dVol.
\end{align*}


It follows from this that $\omega$ is not exact: indeed, if $\omega = d\eta$, then $d(\omega^n) = d(\eta
\wedge d\eta \wedge \dots \wedge d\eta)$, so Stokes theorem would give $\int_X \omega^n = 0$. The injective
morphism can be given as:
\begin{align*}
k[x]/x^{n+1} &\to H^*(X,\R) ,\\
x &\mapsto [\omega].
\end{align*}
In particular, $b_2(X) \geq 1$, $[\omega]$ being a nontrivial element. All spheres except $S^2$ have $H^2(S^i, \R) = 0$,
so they cannot admit K\"{a}hler structures.


\subsection*{Exercise 3.1.12 in \cite{Huy}}
\emph{Let $X$ be a complex manifold endowed with a K\"{a}hler form and let $\alpha$ be
a closed (1,1)-form which is primitive (at every point in the sense of Section 1.2).
Show that $\alpha$ is harmonic, i.e. $\Delta(\alpha) = 0$.}
\vspace{3mm}

If $\alpha \in \mathcal{A}^{1,1}$, then $\p \alpha \in \mathcal{A}^{2,1}$ and $\bar \p \alpha \in \mathcal{A}^{1,2}$.
Since $\p \alpha + \bar \p \alpha = d\alpha = 0$, and $\p \alpha, \bar \p \alpha$ are in different orthogonal components
of the decomposition $\mathcal{A}^3 = \bigoplus_{p+q = 3} \mathcal{A}^{p,q}(X)$, it must be the case that
$\p \alpha = \bar \p \alpha = 0$.

Then applying K\"{a}hler identities we have:
\[  \p^*\alpha = i[\Lambda,\bar \p] \alpha = i\Lambda \bar \p \alpha - i\bar \p \Lambda \alpha = 0.	\]
Since $\p \alpha = \p^*\alpha = 0$, it follows that $\Delta_{\p} \alpha = 0$, so $\Delta \alpha = 0$.


\subsection*{Exercise 3.2.1 in \cite{Huy}}
\emph{Let $(X, g)$ be a Kahler manifold. Show that the Kahler form $\omega$ is harmonic.}
\vspace{3mm}

We use the K\"{a}hler identity $[L,\Delta] = 0$, which gives:
\[	\Delta(\omega) = \Delta L(1) = L \Delta (1) = 0.	\]


\subsection*{Exercise 3.2.4 in \cite{Huy}}
\emph{Recall Exercise 2.6.11 and show that on a compact K\"{a}hler manifold ``the
limit $\lim_{t\to 0}$ commutes with hypercohomology'', i.e.}
\[	\lim_{t\to 0} \HH^k\big(X, (\Omega_X^{\bullet}, td)\big) = \HH^k\big(X, \lim_{t\to 0}(\Omega_X^{\bullet}, td)\big).	\]
\vspace{3mm}

According to Exercise 2.6.11, we have:\todo{do 2.6.11}
\begin{align*}
&\HH^k\big(X, (\Omega_X^{\bullet}, td)\big) = H^k(X,\C) \\
&\HH^k\big(X, \lim_{t\to 0}(\Omega_X^{\bullet}, td)\big) = \bigoplus_{p+q = k} H^{p,q}(X).
\end{align*}
The Hodge decomposition implies that the RHS are equal, so the quantities on the left must also be equal.


\subsection*{Exercise 3.2.5 in \cite{Huy}}
\emph{Show that for a complex torus of dimension one the decomposition in
Corollary 3.2.12 does depend on the complex structure. It suffices to consider $H^1$.}
\vspace{3mm}

Using Dolbeault cohomology, we obtain natural isomorphisms $H^{p,q}(X) \cong H^q(X,\Omega_X^q)$. 
In particular, for $X$ a 1-dimensional torus, the Hodge decomposition is:
\[	H^1(X,\C) \cong H^0(X,\Omega_X) \oplus H^1(X,\mathcal{O}_X).	\]
The global holomorphic forms depend on the complex structure of $X$, since they're spanned over $\C$ by the form
$dz$ mod $\Lambda$, where $\Lambda$ is the lattice giving the complex structure. (Similarly, the sheaf of holomorphic
functions $\mathcal{O}_X$ clearly depends on the complex structure. It may be less clear that its first cohomology
also does.)

The moral of the story is: as one varies the complex structure on $X$ while preserving the topology, $H^k(X,\C)$ remains
unchanged, but the factors in the Hodge decomposition change. This leads to the study of variational Hodge structures.


\subsection*{Exercise 3.2.6 in \cite{Huy}}
\emph{Show that the odd Betti numbers $b_{2i+1}$ of a compact K\"{a}hler manifold are
even.}
\vspace{3mm}

We use the Hodge decomposition:
\[	H^{2i+1}(X,\C) = \bigoplus_{p+q=2i+1} H^{p,q}(X).	\]
Since $\overline{H^{p,q}(X)} = H^{q,p}(X)$, which follows from the analogous statement for harmonic forms,
$\dim_{\C} H^{p,q}(X) = \dim_{\C} H^{q,p}(X)$. Therefore:
\[	b_{2i+1}(X) = \sum_{p=0}^i 2 \dim_{\C} H^{p,2i+1-p}.	\]


\subsection*{Exercise 3.2.7 in \cite{Huy}}
\emph{Are Hopf surfaces (cf. Section 2.1) K\"{a}hler manifolds?}
\vspace{3mm}

Hopf surfaces $X$ are diffeomorphic to $S^1 \times S^3$, so in particular $H^1(X,\C) \cong \C$. Using the result of
Exercise 3.2.6, Hopf surfaces don't admit K\"{a}hler structures.


\subsection*{Exercise 3.2.8 in \cite{Huy}}
\emph{Show that holomorphic forms, i.e. elements of $H^0(X, \Omega^p)$, on a compact
K\"{a}hler manifold $X$ are harmonic with respect to any K\"{a}hler metric.}
\vspace{3mm}

From Dolbeault cohomology we have a canonical isomorphism $H^0(X,\Omega^p) \cong H^{p,0}(X)$, i.e. all global
sections $\omega_p$ of $\Omega^p$ are representatives for $H^{p,0}(X)$. In particular, they are $\bar \p$-closed.
Using the Hodge decomposition for $\bar \p$, a $\bar \p$-closed form is an element of $\bar \p \mathcal{A}^{p,q-1} 
+ \mathcal{H}^{p,q}$. Forms in the image of $\bar \p$ are never holomorphic, so we must have $\omega_p
\in \mathcal{H}^{p,q}$.




\subsection*{Exercise 3.2.10 in \cite{Huy}}
\emph{Let $(X, g)$ be a compact hermitian manifold. Show that any $d$-harmonic
(p,q)-form is also $\bar \p$-harmonic.}
\vspace{3mm}

As in the proof of Lemma 3.2.5, we write:
\[	(\Delta(\alpha), \alpha) = ||d\alpha||^2 + ||d^*\alpha||^2.	\]
This shows that $\Delta(\alpha) = 0$ iff $d\alpha = 0$ and $d^*\alpha = 0$. Now for a pure (p,q) form, this is
in turn equivalent to $\p \alpha = \bar \p \alpha = \p^* \alpha = \bar \p^* \alpha = 0$. Applying Lemma 3.2.5,
this is equivalent to $\Delta_{\p}(\alpha) = \Delta_{\bar \p}(\alpha) = 0$. We have proved that $\alpha$ is $d$-harmonic
iff it is both $\p$-harmonic and $\bar \p$-harmonic.



\subsection*{Exercise 3.2.11 in \cite{Huy}}
\emph{Show that $H^{pq}(\PP^n) = 0$ except for $p = q < n$. In the latter case, the space
is one-dimensional. Use this and the exponential sequence to show that $\Pic(\PP^n) = \Z$.}
\vspace{3mm}
 
From singular homology we know that:
\[	H^k(\PP^n,\C) = \left\{ \begin{array}{l} \C, k \text{ even}, \\ 0, k \text{ odd}. \end{array}\right. 	\]
Then for each fixed $p+q=2k$, only one $H^{p,q}(X)$ is nonzero, and that one is 1-dimensional. It suffices to 
show that $H^{k,k}(X)$ is nonzero by exhibiting a nonzero element in it. The K\"{a}hler form $\omega$ is
of type (1,1), so $\omega^k \in \mathcal{A}^{k,k}$. Since $d\omega = 0$, $d(\omega^k) = 0$, so $\omega^k
\in H^{k,k}(X)$, and since $\omega^k$ cannot be exact\footnote{See solution to Exercise 3.1.8.} is a nonzero element.

A consequnece is that $H^k(\PP^n,\mathcal{O}_X) \cong H^{0,k} = 0$ for all $k>0$. The long exact sequence associated to
the exponential sequence gives the exactness of:
\[	0 \to H^1(\PP^n,\mathcal{O}_{\PP^n}^{\times}) \to H^2(\PP^n,\Z) \to 0.	\]
That is, $\Pic(\PP^n) \cong \Z$.



\subsection*{Exercise 3.2.13 in \cite{Huy}}
\emph{Prove the $dd^c$-lemma: If $\alpha \in\mathcal{A}^k(X)$ is a $d^c$-exact and $d$-closed form
on a compact K\"{a}hler manifold $X$ then there exists a form $\beta \in\mathcal{A}^{k-2}(X)$ such that
$\alpha = dd^c \beta$. (A proof will be given in Lemma 3.A.22.)}
\vspace{3mm}

Let $\alpha = d^c \gamma$, and we use the Hodge decomposition for the $d$ operator. $\gamma = d\beta + \delta + d^*\eta$,
with $\delta$ harmonic. On a K\"{a}hler manifold, $\Delta (\delta) = 0$ implies $\Delta_{\p}(\delta) = \Delta_{\bar \p}
(\delta) = 0$.\footnote{Note that, in the case $X$ hermitian but not K\"{a}hler, Exercise 3.2.10 gives the same
statement, but only for pure forms. Here we don't make any assumption about purity.} It follows that $\bar \p(\delta)
= \p(\delta) = 0$, so $d^c(\delta) = 0$. We arrive at:
\[	\alpha = d^c d(\beta) + d^cd^*\eta.	\]
To show that the second term is 0, use the fact that $dd^c = -d^cd$. Then:
\[	0 = d\alpha = 0 + dd^c d^*\eta = - dd^*d^c \eta.	\]
It follows that:
\[	0 = (dd^*d^c \eta, d^c \eta) = ||d^*d^c \eta||^2,	\]
so $d^*d^c \eta = 0$. Finally, $\alpha = d^cd(\beta) = dd^c(-\beta)$.


\subsection*{Exercise 3.2.15 in \cite{Huy}}
\emph{Let $(X,g)$ be a compact hermitian manifold and let $[\alpha] \in H^{pq}(X)$. Show
that the harmonic representative of $[\alpha]$ is the unique $\bar \p$-closed form with minimal
norm $||\alpha||$. (This is the analogue of Lemma A.0.18.)}
\vspace{3mm}

If we impose that $\bar \p \alpha = 0$, from the Hodge decomposition for $\bar \p$ we obtain $\alpha = \bar \p \beta
+ \gamma$, with $\gamma$ harmonic. Since $(\bar \p \beta, \gamma) = 0$:
\[	||\alpha||^2 = ||\bar \p \beta||^2 + ||\gamma||^2.	\]
Hence minimizing $||\alpha||$ is equivalent to $\bar \p \beta = 0$. We obtain $\alpha = \gamma$, and Corollary 3.2.9
guarantees that the harmonic representative $\gamma$ is unique for its cohomology class.


\subsection*{Exercise 3.2.16 in \cite{Huy}}
\emph{Let $X$ be a compact K\"{a}hler manifold. Show that for two cohomologous
K\"{a}hler forms $\omega$ and $\omega'$, i.e. $[\omega] = [\omega'] \in H^2(X, \R)$, there exists a real function $f$ such
that $\omega' = \omega + i\p \bar\p f$.}
\vspace{3mm}

$\omega' - \omega \in \mathcal{A}^{1,1}(X,\R)$ is $d$-exact, so the $\p \bar \p$ lemma implies $\omega' - \omega
= \p \bar \p g$ for some $g \in \mathcal{A}^{0,0}(X,\C)$. Note that, locally:
\[	\p \bar \p g = \sum_{i,j = 1}^n \frac{\p^2 g}{\p z_i \p \bar z_j} dz_i \wedge d\bar z_j.	\]
The requirement that $\p \bar \p g$ be real translates into $\frac{\p^2 g}{\p z_i \p \bar z_j}$ real for all $i,j$, so
$g=if$, with $f$ real.\todo{figure out how this actually works}


\subsection*{Exercise 3.3.1 in \cite{Huy}}
\emph{Go back to the proof of Lemma 3.3.1 and convince yourself that we have
not actually used that $X$ is K\"{a}hler. This comes in when considering the bidegree
decomposition of $H^k$.}
\vspace{3mm}

Indeed, the proof of Lemma 3.3.1 does not use the fact that $\mathcal{H}_{\p}^{p,q}$ and $\mathcal{H}_{\bar \p}^{p,q}$
coincide, nor the bidegree decomposition on cohomology. The lemma should be read as proving the existence of a projection map
$H^k(X,\C) \to H^{0,k}(X)$, irrespective of whether the bidegree decomposition $H^k(X,\C) = \bigoplus_{p+q = k} H^{p,q}(X)$
exists.

Tracing the steps of the proof, we need the existence and uniqueness of a $d$-harmonic representative $\alpha$ for $[\alpha]
\in H^k(X,\C)$; this comes from the Hodge decomposition for the $d$ operator. (See Theorem A.0.16.) The (0,k) 
part of $\alpha$ is also $d$-harmonic, since $\Delta_d$ preserves the bidegree decomposition of forms. Due to
Exercise 3.2.10, pure $d$-harmonic forms are also $\bar \p$-harmonic. Hence $\Pi^{0,k}(\alpha)$ is $\bar \p$-harmonic.
Therefore $\Pi^{0,k}$ induces a map $H^k(X,\C) \to H^{0,k}(X)$, which is surjective, due to the Hodge decomposition for the
$\bar \p$ operator and Corollary 3.2.9.


\subsection*{Exercise 3.3.2 in \cite{Huy}}
\emph{Let $X$ be a Hopf surface. Show that the Jacobian, i.e. $H^1(X, \mathcal{O}_X)/H^1(X, \Z)$,
is not a compact torus in a natural way. In fact, $H^1(X, \Z) = \Z$ and $H^1(X, \mathcal{O}_X) = \C$.}
\vspace{3mm}

Since $X$ is diffeomorphic to $S^1 \times S^3$, singular cohomology gives $H^1(X,\Z) \cong \Z$. Then the
Universal coefficient theorem for cohomology gives $H^1(X,\C) \cong \C$, and Exercise 3.3.1 asserts the existence
of a surjective map $H^1(X,\C) \to H^1(X,\mathcal{O}_X)$. Therefore $H^1(X,\mathcal{O}_X)$ is either $0$ or $\C$.
Consider, then, the long exact sequence on cohomology associated to the exponential sequence. For compact
complex manifolds it splits into a short
exact sequence:
\[	0 \to \Z \to \C \to \C^{\times} \to 0,	\]
followed by a long exact sequence:
\[	0 \to H^1(X,\Z) \to H^1(X,\mathcal{O}_X) \to \dots.	\]
In particular, $H^1(X,\Z) \to H^1(X,\mathcal{O}_X)$ is injective, so the only option is $H^1(X,\mathcal{O}_X) \cong \C$.
\footnote{As a side-note, this also proves that Hopf surfaces have no global holomorphic forms.}
$\Pic^0(X)$ is then isomorphic to the quotient $\C/\Z \cong \C^{\times}$.


\subsection*{Exercise 3.3.4 in \cite{Huy}}
\emph{Let $X$ be a K\"{a}hler surface with $\kod(X) = -\infty$. Show that its signature is $\big(1,h^{1,1}(X)-1\big)$.}
\vspace{3mm}

Recall that $\kod(X)= \trdeg_{\C} Q(R(X)) - 1$, or $\kod(X) = -\infty$ if $R(X) \cong \C$. By hypothesis we are in the
latter case, so $H^0(X,K_X^{\otimes m}) = 0$ for all $m>0$. In particular, taking $m=1$ gives
$0 = H^0(X,\Omega_X^{\wedge 2}) = H^{2,0}(X)$. The Hodge index theorem then implies that $\sgn(X) = \big(1,h^{1,1}(X)-1\big)$.


\subsection*{Exercise 3.3.5 in \cite{Huy}}
\emph{Let $X$ be a compact K\"{a}hler manifold of dimension $n$ and let $Y \subset X$ be
a smooth hypersurface such that $[Y] \in H^2(X, \R)$ is a Kahler class. Show that the
canonical restriction map $H^k(X, \R) \to H^k(Y, \R)$ is injective for $k \leq n — 1$. (This is
one half of the so-called weak Lefschetz theorem (cf. Proposition 5.2.6). It can be
proven by using Poincare duality on $X$ and the Hard Lefschetz theorem 3.3.13).}
\vspace{3mm}

Let $\alpha \in H^k(X,\R)$ with $0\leq k \leq n-1$. Due to the Hard Lefschetz Theorem, if $\alpha \neq 0$, then
$0 \neq [Y] \smile \alpha  \in H^{k+2}(X,\R)$. Using Poincar\'{e} duality with coefficients in the field $\R$, 
the cup product pairing is nondegenerate, so there exists $\gamma \in H^{2n-k-2}$ such that
$\int_X [Y] \smile \alpha \smile \gamma = 1$. By the definition of the fundamental class $[Y]$, this implies
$\int_Y \alpha|_Y \smile \gamma|_Y = 1$. In particular, $\alpha|_Y \neq 0$.\footnote{The fact that this result,
as well as the Hard Lefschetz Theorem, are stated over $\R$ and not $\C$ makes them stronger. The complexified
version is obtained by, well, complexifying both sides.}




\subsection*{Exercise 3.3.6 in \cite{Huy}}
\emph{Construct a complex torus $X = \C^2/\Gamma$ such that $NS(X) = 0$. Conclude that
such a torus cannot be projective and, moreover, that $K(X) = \C$. Observe that in
any case $\Pic(X) \neq 0$.}
\vspace{3mm}

Let $X = V/\Gamma$ be the complex torus, with $V \cong \C^2$. Then $H^{1,0}(X) = H^0(X,\Omega_X) \cong V^*$, since
the cotangent bundle of $X$ is trivial.
Then $H^{0,1}(X) \cong \bar V^*$, and $H^1(X,\Z) \cong \Gamma^*$. $\Gamma^*$ sits in both $V^*$ and $\bar V^*$,
and in particular sits in their direct sum $V^* \oplus \bar V^* \cong H^1(X,\C)$. Choosing a basis $e_1, \dots,
e_4$ for $\Gamma^*$ and a basis $dz_1, dz_2$ for $V^*$ gives the following embedding:
\begin{align}
\label{eq:latticeh1}
\begin{split}
\Gamma^* &\hookrightarrow V^* \oplus \bar V^* \\
e_1 &\mapsto a_1 dz_1 + a_2 dz_2 + \bar a_1 d\bar z_1 + \bar a_2 d\bar z_2 \\
e_2 &\mapsto b_1 dz_1 + b_2 dz_2 + \bar b_1 d\bar z_1 + \bar b_2 d\bar z_2 \\
e_3 &\mapsto c_1 dz_1 + c_2 dz_2 + \bar c_1 d\bar z_1 + \bar c_2 d\bar z_2 \\
e_4 &\mapsto f_1 dz_1 + f_2 dz_2 + \bar f_1 d\bar z_1 + \bar f_2 d\bar z_2 .
\end{split}
\end{align}
Here $(a_1,a_2), \dots, (f_1,f_2)$ are the elements of $\C^2$ which give the embedding $\Gamma \hookrightarrow V$
which defines the complex torus. We interpret \ref{eq:latticeh1} as giving the embedding $H^1(X,\Z) \to H^1(X,\C)$.
But recall that, since the $T*X$ is trivial, we have canonical isomorphisms:
\begin{align*}
H^2(X,\Z) &\cong \bigwedge^2 H^1(X,\Z) \cong \Gamma^* \wedge \Gamma^*, \\
H^2(X,\C) &\cong \bigwedge^2 H^1(X,\C) \cong V^* \wedge \bar V^*, \\
H^{p,q}(X) &\cong \bigwedge^p H^{1,0}(X) \wedge \bigwedge H^{0,1}(X) \cong \bigwedge^p V^* \wedge \bigwedge^q \bar V^*.
\end{align*}
Due to \ref{eq:latticeh1}, the inclusion $H^2(X,\Z) \to H^2(X,\C)$ then takes the form:
\begin{align}
\label{eq:latticeh2}
\begin{split}
e_1 \wedge e_2 \mapsto& (a_1 b_2 - a_2 b_1) dz_1 \wedge dz_2 + (a_1 \bar b_1 - \bar a_1 b_1) dz_1 \wedge d\bar z_1
+ (a_1 \bar b_2 - \bar a_2 b_1) dz_1 \wedge d\bar z_2 \\
+& (a_2 \bar b_1 - \bar a_1 b_2) dz_2 \wedge d\bar z_1
+ (a_2 \bar b_2 - \bar a_2 b_2) dz_2 \wedge d\bar z_2 + (\bar a_1 \bar b_2 - \bar a_2 \bar b_1) d\bar z_1 \wedge d\bar z_2,
\end{split}
\end{align}
and analogously for the other 5 standard basis elements of $\Gamma^* \wedge \Gamma^*$. Note that the first term in
\ref{eq:latticeh2} belongs to $H^{2,0}(X)$, the next four to $H^{1,1}(X)$, and the last one to $H^{0,2}(X)$.

The condition $NS(X) = 0$ translates into the fact that the images of integral linear combinations:
\[	n_{12} e_1 \wedge e_2 + n_{13} e_1 \wedge e_3 + n_{14} e_1 \wedge e_4 + n_{23} e_2 \wedge e_3 +
n_{24} e_2 \wedge e_4 + n_{34} e_3 \wedge e_4	\]
under the map \ref{eq:latticeh2} intersect the subspace $V^* \wedge \bar V^* \subset (V^* \oplus \bar V^*) \wedge
(V^* \oplus \bar V^*)$ trivially. This is equivalent to the requirement that the $V^* \wedge V^*$ component does not
vanish, i.e.:
\begin{equation}
\label{eq:ns0cond}
	n_{12} (a_1 \bar b_2 - \bar a_2 b_1) + n_{13} (a_1 \bar c_2 - \bar a_2 c_1) + n_{14} (a_1 \bar f_2 - \bar a_2 f_1)
 + n_{23} (b_1 \bar c_2 - \bar b_2 c_1) +
n_{24} (b_1 \bar f_2 - \bar b_2 f_1) + n_{34} (c_1 \bar f_2 - \bar c_2 f_1) \neq 0.
\end{equation}
(The condition for $H^{0,2}$ is just the conjugate of this one.) We use \ref{eq:ns0cond} to exhibit and example and a non-example.
Take the complex torus with lattice spanned by:
\[	(1,0) \hspace{5mm} (i b_1, b_2) \hspace{5mm} (0,1) \hspace{5mm} (f_1,i f_2),	\]
and all unspecified coefficients are real. Then \ref{eq:ns0cond} becomes:
\[	n_{12} b_2 - n_{13} + i n_{14} f_2 + i n_{23} b_1 - i n_{24} (b_1 f_2 + b_2 f_1) - n_{34} f_1 \neq 0.	\]
To satisfy this, it suffices to choose the remaining coefficients such that $b_2, f_1, b_2/f_1$ are irrational, and
similarly $b_1, f_2, b_1f_2 + b_2f_1$ and all their pairwise quotients are irrational. In particular, the complex torus
given by the lattice:
\[	(1,0) \hspace{5mm} (i \sqrt{2}, \sqrt{3}) \hspace{5mm} (0,1) \hspace{5mm} (\sqrt{5},i \sqrt{7})	\]
has $NS(X) = 0$. For a non-example, take the torus given by the lattice:
\[	(1,0) \hspace{5mm} (i, 0) \hspace{5mm} (0,1) \hspace{5mm} (0,i). \]
In this case, the $H^{2,0}$ component vanishes as soon as $n_{13} - n_{24}= 0$ and $n_{14} + n_{23} = 0$, which gives
$\rk NS(X) = 4$. By playing around with the coefficients, one can construct complex tori for all values $0 \leq \rk NS(X)
\leq 4$.

In the case $NS(X) = 0$, the image of $\Pic(X)$ in $H^2(X,\Z) = 0$, so $\Pic(X) = \Pic^0(X)$. However, if $X$ is projective,
letting $\phi : X \to \PP^N$ denote a projective embedding, $c_1(\phi^*\mathcal{O}_{\PP^N}(1)) = \phi^* c_1(
\mathcal{O}_{\PP^N}(1)) \neq 0$. \todo{why is this?} This contradicts $\Pic(X) = \Pic^0(X)$.


\subsection*{Exercise 3.3.7 in \cite{Huy}}
\emph{Show that any complex line bundle on $\PP^n$ can be endowed with a unique
holomorphic structure. Find an example of a compact (K\"{a}hler, projective) manifold
and a complex line bundle that does not admit a holomorphic complex structure.}
\vspace{3mm}

First, every complex line bundle $E$ on $\PP^n$ admits a holomorphic structure. This is because $H^2(\PP^n,\Z) = 
H^{11}(\PP^n,\Z)$, so $c_1(E) \in H^{11}(\PP^n,\Z)$, which is $c_1(\Pic(\PP^n))$,
due to the Lefschetz theorem on (1,1) classes. Then choose any $L\in \Pic(\PP^n)$ such that $c_1(L) = c_1(E)$, and
we claim that $L$ is a holomorphic line bundle which is isomorphic to $E$ as a complex line bundle. To see this, it
suffices to prove that complex line bundles are uniquely determined by their first Chern class. Consider the (smooth)
exponential sequence:
\[	0 \to \Z \to \mathcal{A}^0_X \to {\mathcal{A}^0_X}^{\times} \to 0,	\]
for sheaves of $C^{\infty}$ complex valued functions, rather than holomorphic functions. Then $H^i(X,\mathcal{A}^0_X) = 0$
for $i>0$, because $\mathcal{A}^0_X$ is soft, due to the existence of partitions of unity. Hence the associated long
exact sequence on cohomology splits into isomorphisms, and in particular:
\begin{equation}
\label{eq:complexlb}
0 \to 	H^1(X,{\mathcal{A}^0_X}^{\times}) \overset{c_1}{\to} H^2(X,\Z) \to 0.
\end{equation}
The LHS parametrizes isomorphism classes of complex vector bundles, so the claim is proved.

It remains to see that, for $X = \PP^n$, the holomorphic structure $L$ on the complex vector bundle $E$ is unique. From
the exactness of the (holomorphic) exponential sequence, we obtain:
\[	0 \to \Pic^0(X) \to \Pic(X) \overset{c_1}{\to} H^2(X,\Z), 	\]
where $\Pic^0(X) = H^{01}(X,\C) / H^1(X,\Z)$. Hence $\Pic^0$ parametrizes holomorphic structures on a given complex
vector bundle $E$. For $X = \PP^n$, $H^{01}(\PP^n) = 0$, so $\Pic^0(\PP^n) = 0$. 

Compact K\"{a}hler manifolds $X$ support complex vector bundles $E$ with no holomorphic structure if and only if
$\im(H^2(X,\Z) \to H^2(X,\C)) \not \subset H^{11}(X)$. A necessary condition for this is that $h^{20}(X) \neq 0$.
This is satisfied, for example, by a complex torus $X =\C^2/\Gamma$, which has $h^{20} = 1$, $h^{11} = 4$, $h^{02} = 1$.
We can choose the lattice $\Gamma$ such that $X$ is an abelian variety, which is projective (see Exercise 3.3.6).
In this case:
\[	H^{2,0}(X,\Z) \cong \Z, \hspace{5mm} H^{1,1}(X,\Z) \cong \Z^4, \hspace{5mm} H^{0,2}(X,\Z) \cong \Z.	\]
An element $c \in H^{2,0}(X,\Z)$ is not in the image of $\Pic(X)$, so it cannot be the Chern class of a holomorphic
line bundle. However, using \ref{eq:complexlb}, $c$ is the Chern class of a complex line bundle.\footnote{Note that,
if we choose a generic lattice $\Gamma$, $H^{2,0}(X,\Z)$, $H^{1,1}(X,\Z)$, $H^{0,2}(X,\Z)$ are all 0. The argument
still applies.}


\subsection*{Exercise 3.3.8 in \cite{Huy}}
\emph{Show that on a complex torus $\C^n/\Gamma$ the trivial complex line bundle admits
many (how many?) holomorphic non-trivial structures.}
\vspace{3mm}

The trivial complex line bundle $E$ satisfies $c_1(E) = 0$, so in particular $c_1(E) \in H^{1,1}(X,\Z)$. Therefore
it admits holomorphic structures. These are parametrized by $\Pic^0(X)$. (See solution to exercise 3.3.7.) Using Theorem
3.3.6, $\Pic^0(\C^n/\Gamma)$ is a complex torus of complex dimension $b_1(\C^n/\Gamma)/2$. Since the (smooth) cotangent bundle of a torus
is trivial, we obtain representatives in deRham cohomology for $H^i(\C^n/\Gamma,\C)$ as follows:
\begin{equation}
\label{eq:combderham}
	H^i(X , \C) \cong \bigwedge^i H^0(X, T*X) \cong \bigwedge^i \C^{2n}.
\end{equation}
In particular, $b_1(X) = \binom{2n}{1} = 2n$, so $\Pic^0(X)$ is a n-complex dimensional torus.\footnote{See Exercise
3.3.10 below for an argument that this torus is dual to the original one.}

We can say more. We can obtain representatives in Dolbeault cohomology for $H^{pq}(X)$ as follows.
\[	H^{pq}(X, \C) \cong \bigwedge^p H^0(X, T^{*(1,0)}X) \wedge \bigwedge^q H^0(X, T^{*(0,1)}X) \cong 
\bigwedge^p \C^n \wedge \bigwedge^q \C^{n}. 	\]
In particular, $h^{pq}(X) = \binom{n}{p} + \binom{n}{q}$. Then:
\[	b_{i}(X) = \sum_{p+q=i} \binom{n}{p} \binom{n}{q} = \binom{2n}{i},	\]
which agrees with \ref{eq:combderham}.


\subsection*{Exercise 3.3.9 in \cite{Huy}}
\emph{Let $X$ be a compact K\"{a}hler manifold. Show that the fundamental class
$[Y] \in H^{p,p}(X)$ of any compact complex submanifold $Y \subset X$ of codimension $p$ is
non-trivial. Is this true for Hopf manifolds?}
\vspace{3mm}

Let $\omega$ be a K\"{a}hler form on $X$. Then $\omega|_Y$ is a K\"{a}hler form on $Y$. In particular, 
$\omega|_Y^{n-p}$ is a volume form on $Y$, and we have\footnote{See Exercise 3.1.8}:
\[	(n-p)! =  \int_Y \omega|_Y^{n-p} = \int_X [Y] \smile \omega|_Y^{n-p}.	\]
Therefore $0 \neq [Y] \in H^{p,p}(X)$.\footnote{For $X$ non-K\"{a}hler, compact complex submanifolds also have
volume forms, but they are not necessarily restrictions of forms on $X$.}

If, instead, $X$ is a Hopf manifold of dimension $n\geq 2$, then it is diffeomorphic to $S^1 \times S^{2n-1}$.
The latter can be given a CW structure with only one cell in the dimensions 0, 1, 2n-1, 2n. Therefore
$H^2(X,\C) = 0$. So every hypersurface $Y$ has fundamental class 0. Note that, a priori, this statement could
be vacuous: non-algebraic complex manifolds tend to have few submanifolds. However, Hopf surfaces do contain elliptic
curves (see Exercise 2.1.7).


\subsection*{Exercise 3.3.10 in \cite{Huy}}
\emph{Let $X$ be a complex torus $\C^n/\Gamma$. Show that $\Pic^0(\Pic^0(X))$ is naturally isomorphic to $X$.}
\vspace{3mm}

Recall the setup at the beginning of Exercise 3.3.6. If $X = V/\Gamma$, where $\dim_{\C} V = n$, then there are natural
isomorphisms $H^{0,1}(X) \cong \bar V^*$ and $H^1(X,\Z) \cong \Gamma^*$. Then, by definition:
\[	\Pic^0(X) \cong \bar V^* / \Gamma^* .	\]
Applying the same procedure again:
\[	\Pic^0(\Pic^0(X)) \cong \overline{(\bar V^*)^*} / (\Gamma^*)^* \cong V/\Gamma = X.	\]


\subsection*{Exercise 3.3.11 in \cite{Huy}}
\emph{Show that $\Alb(X)$ and $\Pic^0(X)$ of a compact K\"{a}hler manifold $X$ are dual
to each other, i.e. $\Pic^0(\Alb(X)) \cong \Pic^0(X)$.}
\vspace{3mm}

By definition, $\Alb(X) = H^0(X,\Omega_X)^* / H_1(X,\Z)$, which is a torus of dimension $n = \dim_C H^0(X,\Omega_X)$.
Using again the setup at the beginning of Exercise 3.3.6, $H_1(\Alb(X),\Z)$ is naturally isomorphic to the lattice
of $\Alb(X)$, which in this case is $H_1(X,\Z)$. Hence $H_1(\Alb(X),\Z) \cong H_1(X,\Z)$. Moreover,
$H^0(\Alb(X), \Omega_{\Alb(X)}) \cong H^0(X,\Omega_X)$ naturally due to Proposition 3.3.8, which is a simple consequence
of the definition of the Albanese map. Both $X$ and $\Alb(X)$ are K\"{a}hler manifolds, so we have the following
commutative diagrams, where all maps are natural isomorphisms.
\[
\begin{tikzcd}
H_1(\Alb(X),\Z)*\arrow{r}\arrow{d} & H^1(\Alb(X),\Z)\arrow{d} & 
\overline{H^0(\Alb(X), \Omega_{\Alb(X)})}\arrow{r}\arrow{d} & H^1(\Alb(X), \mathcal{O}_{\Alb(X)})\arrow{d} \\
H_1(X,\Z)*\arrow{r} & H^1(X,\Z) &
\overline{H^0(X, \Omega_{X})}\arrow{r} & H^1(X, \mathcal{O}_{X})
\end{tikzcd}
\]
In particular, the naturality of the right column in both diagrams gives the following commutative diagram,
where $i_X:\Z \to \mathcal{O}_X$, $i_{\Alb(X)} : \Z \to \mathcal{O}_{\Alb(X)}$ are inclusions.
\[
\begin{tikzcd}
H^1(\Alb(X),\Z) \arrow{r}{i_*}\arrow[swap]{d}{\cong} & H^1(\Alb(X), \mathcal{O}_{\Alb(X)})\arrow{d}{\cong} \\
H^1(X,\Z) \arrow{r}{i_*} & H^1(X, \mathcal{O}_{X})
\end{tikzcd}
\]
This implies:
\[	\Pic^0(X) = H^1(X, \mathcal{O}_{X}) /H^1(X,\Z) \cong H^1(\Alb(X), \mathcal{O}_{\Alb(X)}) / H^1(\Alb(X),\Z) 
= \Pic^0(\Alb(X)). 	\]



\section{Huybrechts, Chapter 4}

\subsection*{Exercise 4.1.2 in \cite{Huy}}
\emph{Let $L$ be a holomorphic line bundle of degree $d>2g(C)-2$ on a compact curve $C$. Show that $H^1(C,L) = 0$.
Here, for our purpose, we define the genus $g(C)$ by the formula $\deg(K_X) = 2g(C) - 2$.}

\emph{In other words, $H^1(C,K_X\otimes L) = 0$ for any holomorphic line bundle $L$ with $\deg(L)>0$. In this form,
the statement will be later generalized to the Kodaira vanishing theorem for arbitrary compact K\"{a}hler manifolds.}
\vspace{3mm}

Using Serre duality, we have $H^1(C,L) \cong H^0(C,L^{-1}\otimes \omega_C)^*$. The latter is 0, because $\deg L^{-1}
\otimes \omega_C = 2g(C) - 2 - \deg L < 0$, and the degree of a vector bundle $V$ satisfies:
\[	\deg V = \max \{x| H^0\big(V \otimes_{\mathcal{O}_C} \mathcal{O}_C(-x)\big) \neq 0\}.	\]


\subsection*{Exercise 4.1.3 in \cite{Huy}}
\emph{Show, e.g. by writing down an explicit basis, that}
\[	h^n(\PP^n, \mathcal{O}(k)) = \left\{ \begin{array} {ll} 0 & k>-n-1, \\ \binom{-k-1}{-n-1-k} & k\leq -n-1. \end{array} \right.	\]
\vspace{3mm}

Using Serre duality, together with the fact that $\omega_{\PP^n} \cong \mathcal{O}_{\PP^n}(-n-1)$:
\[	h^n\big(\PP^n,\mathcal{O}(k)\big) = h^0\big(\PP^n,\mathcal{O}(-k)\otimes \mathcal{O}(-n-1)\big) = h^0\big(\PP^n,
\mathcal{O}(-k-n-1)\big).	\]
It follows that this number is 0 if $k>-n-1$, and otherwise equal to the dimension of the $-k-n-1$ graded piece of
$\C[x_0, \dots, x_n]$. By a stars and bars argument, this is $\binom{-k-1}{-n-1-k}$.


\subsection*{Exercise 4.2.6 in \cite{Huy}}
\emph{Let $\nabla$ be a connection on $E$. Describe the induced connections on $\bigwedge^2 E$ and $\det(E)$.}
\vspace{3mm}

Since $s_1 \wedge s_2 = s_1 \otimes s_2 - s_2 \otimes s_1$\footnote{A different convention uses a factor of
$1/k!$ in the definition of $s_1\wedge \dots \wedge s_k$.}, we use the linearity of $\nabla$ to obtain:
\begin{align*}
	\nabla(s_1 \wedge s_2) &= \nabla(s_1 \otimes s_2) - \nabla (s_2\otimes s_1) \\
&= \nabla(s_1)\otimes s_2 + s_1 \otimes \nabla(s_2) - \nabla(s_2) \otimes s_1 - s_2\otimes \nabla(s_1) \\
&= \nabla(s_1)\wedge s_2 + s_1 \wedge \nabla(s_2).
\end{align*}

Similarly, using $s_1\wedge \dots \wedge s_k = \sum_{\sigma \in S^k} (-1)^{\sigma} s_{\sigma(1)}\otimes \dots \otimes 
s_{\sigma(k)}$:
\begin{align*}
\nabla(s_1\wedge \dots \wedge s_k) &= \sum_{\sigma \in S^k}(-1)^{\sigma} \nabla(s_{\sigma(1)}\otimes \dots \otimes 
s_{\sigma(k)}) \\
&= \sum_{\sigma \in S^k}(-1)^{\sigma} \sum_{j=1}^k s_{\sigma(1)}\otimes \dots \otimes \nabla(s_{\sigma(j)}) \otimes
\dots \otimes s_{\sigma(k)} \\
&= \sum_{j=1}^k s_1 \wedge\dots \wedge \nabla(s_j) \wedge \dots \wedge s_k.
\end{align*}
I don't see anything particular about the case $k = \rk(E)$, i.e. $\det(E)$.



\subsection*{Exercise 4.2.7 in \cite{Huy}}
\emph{Show that the pull-back of an hermitian connection is hermitian with respect to the pull-back hermitian structure.
Analogously, the pull-back under a holomorphic map of a connection compatible with the holomorphic structure 
on a holomorphic vector bundle is again compatible with the holomorphic structure on the pull-back bundle.}
\vspace{3mm}

Let $(E,h)$ be a hermitian vector bundle, $\nabla$ a hermitian connection on it, and $s_1, s_2 \in \Gamma(E)$. The
compatibility condition for $\nabla$ is:
\[	d(h(s_1,s_2)) = h(\nabla(s_1),s_2) + h(s_1, \nabla(s_2)).	\]
For any map $\phi : Y \to X$ to the base manifold of $E$, we pull-back both
sides of the equation to get:
\[ d(\phi^*h(\phi^*s_1, \phi^*s_2)) = \phi^*h( \phi^*\nabla(\phi^*s_1), \phi^*s_2) + \phi^*h(\phi^*s_1, \phi^*\nabla(\phi^*s_2)).\]
This proves that $\phi^*\nabla$ satisfies the compatibility condition, as long as we use sections of $\phi^*E$ which are
pull-backs of sections of $E$.

General sections $t_1, t_2 \in \Gamma(\phi^*E)$ are generated as a $C^{\infty}(M)$-module by sections pulled back from $E$.
Hence, using $\C$-linearity of $\nabla$ and $\C$-multilinearity of $h$, we can assume that $t_i = f_i \phi^*(s_i)$. Then
we use the Leibniz rule:
\begin{align*}
\phi^*h( \phi^*\nabla(t_1), t_2) + \phi^*h(t_1, \phi^*\nabla(t_2)) &=
\phi^*h( f_1 \phi^*\nabla(\phi^*s_1) + \phi^*s_1 df_1 , f_2 \phi^*s_2) + 
\phi^*h(f_1 \phi^*s_1, f_2\phi^*\nabla(\phi^*s_2) + \phi^*s_2 df_2) \\
&= f_1 f_2 d(\phi^*h(\phi^*s_1, \phi^*s_2)) + d(f_1f_2) \phi^*h(\phi^*s_1, \phi^*s_2) \\
&= d\big[f_1 f_2 \phi^*h(\phi^*s_1, \phi^*s_2) \big] \\
&= d\big[\phi^*h(t_1,t_2) \big].
\end{align*}

For the second question, let $E$ be a holomorphic vector bundle. Since pullbacks by a holomorphic map respect the 
decomposition $TM = T^{1,0} \oplus T^{0,1}$, we have that
\[	(\phi^* \nabla)^{0,1} = \phi^*(\nabla^{0,1}) = \phi^*(\bar \p_E) = \bar \p_{\phi^* E}.	\]


\subsection*{Exercise 4.2.8 in \cite{Huy}}
\emph{Show that a connection $\nabla$ on a hermitian vector bundle $E$ is hermitian if and only if $\nabla(h) = 0$, where
by $\nabla$ we also denote the naturally induced connection on the bundle $(E\otimes \bar E)^*$.}
\vspace{3mm}

We use Example 4.2.6.iv in \cite{Huy} for the induced connection of a dual bundle. Specifically, if $\nabla$ is
a connection on $E$, $s\in \Gamma(E)$ and $f\in \Gamma(E^*)$, then the induced connection on $E^*$ is:
\[	\nabla(f)(s) = d(f(s)) - f(\nabla(s)).	\]
We apply this to find the induced connection on $(E\otimes \bar E)^*$. Let $s_1 \in \Gamma(E)$, $s_2 \in \Gamma(\bar E)$
and $h\in \Gamma(E\otimes \bar E)^*$. Then:
\begin{align*}
	\nabla(h)(s_1, s_2) &= d(h(s_1,s_2)) - h(\nabla(s_1)\otimes s_2 + s_1 \otimes \nabla(s_1)) \\
&= d(h(s_1,s_2)) - h(\nabla(s_1),s_2) - h(s_1, \nabla(s_2)).
\end{align*}



\subsection*{Exercise 4.4.1 in \cite{Huy}}
\emph{Let $C$ be a connected compact curve. Then there is a natural isomorphism
$H^2(C,\Z) \cong \Z$. Show that with respect to this isomorphism (or, rather, its $\R$-linear
extension) one has $c_i(L) = \deg(L)$ for any line bundle $L$ on $C$.}
\vspace{3mm}

\todo{do this}


\subsection*{Exercise 4.4.2 in \cite{Huy}}
\emph{Show that for a base-point free line bundle $L$ on a compact complex manifold
$X$ the integral $\int_X c_1(L)^n$ is non-negative.}
\vspace{3mm}

Since $L$ is base-point free, it determines a morphism $\phi : X \to \PP^m$ for some $m$, such that $L = \phi^*
\mathcal{O}_{\PP^m}(1)$. Then:
\[	\int_X c_1(L)^n = \int_X f^*\big(c_1(\mathcal{O}_{\PP^m}(1))^n \big).	\]
We distinguish 3 cases:
\begin{enumerate}
\item $m<n$, then $c_1(\mathcal{O}_{\PP^m}(1))^n \in H^{2n}(\PP^m, \C) = 0$, so the integral is 0.
\item $m=n$, in which case $\phi$ is a covering map outside of a set of measure 0. Let $k$ be the degree of the covering.
\[	\int_X c_1(L)^n = k \int_{\PP^n} c_1(\mathcal{O}_{\PP^n}(1))^n = k \int_{\PP^n} \omega_{FS} = k.	\]
We have used Exercise 4.4.7 to identify $c_1(\mathcal{O}_{\PP^n}(1))^n$ with $\omega_{FS}$, and Exercise 3.1.4 to
evaluate the last integral to 1.
\item $m>n$, in which case $\phi = i \circ \pi$ is the composition of a branched covering map $\pi$ and an embedding $i$. The
effect of $\pi$, similarly to the case 2. above, is to multiply the integral by a factor $k>0$, the degree of the covering.
Thus it remains to treat the case of an embedding $\phi : X \to \PP^m$.

Let $x \in \PP^m - \phi(X)$, and let $p_x : \PP^m - \{x\} \to \PP^{m-1}$ be the projection from a point.\footnote{One should
say something about how to choose $x$ such that $\dim(p_x \circ \phi(X)) = \dim(X)$.}
\[	X \overset{\phi}{\to} \phi(X) \overset{p_x}{\to} \PP^{m-1}	\]
Then $p_x$ is linear, so $p_x^*(\mathcal{O}_{\PP^{m-1}}(1)) = \mathcal{O}_{\PP^m}(1)|_{\phi(X)}$.  \todo{finish this}
\end{enumerate}




\subsection*{Exercise 4.4.3 in \cite{Huy}}
\emph{Show that $\td(E_1 \oplus E_2) = \td(E_1) \cdot \td(E_2)$.}
\vspace{3mm}

With $\nabla$ the induced connection on $E_1 \oplus E_2$, we have $F_{\nabla} = F_{\nabla_1} \oplus F_{\nabla_2}$.
Then:
\[	\det\left( t \frac{i}{2\pi} F_{\nabla} \right) = \det\left( t \frac{i}{2\pi} F_{\nabla_1} \right)
\det\left( t \frac{i}{2\pi} F_{\nabla_2} \right),	\]
and
\[	\det\left(I - e^{-t \frac{i}{2\pi}F_{\nabla}}\right) = \det\left(I - e^{-t \frac{i}{2\pi}F_{\nabla_1}}\oplus
e^{-t \frac{i}{2\pi}F_{\nabla_2}}\right) = \det\left(I - e^{-t \frac{i}{2\pi}F_{\nabla_1}}\right)
\det\left(I - e^{-t \frac{i}{2\pi}F_{\nabla_2}}\right).	\]
This gives $\td(E_1 \oplus E_2) = \td(E_1) \cdot \td(E_2)$.



\subsection*{Exercise 4.4.4}
\emph{Compute the Chern classes of (the tangent bundle of) $\PP^n$ and $\PP^n \times \PP^m$.
Try to interpret $\int_{\PP^n}c_n(\PP^n)$ and $\int_{\PP^n\times \PP^m}c_{n+m}(\PP^n\times \PP^m)$.}
\vspace{3mm}

We use the Euler sequence for $\PP^n$:
\[	0 \to \mathcal{O}_{\PP^n} \to \mathcal{O}_{\PP^n}(1)^{\oplus(n+1)} \to \mathcal{T}_{\PP^n} \to 0.	\]
Every short exact sequence of complex vector bundles splits, and that is all we need to apply the Whitney sum property:
\[	c\left(\mathcal{O}_{\PP^n}(1)^{\oplus(n+1)}\right) = c(\mathcal{O}_{\PP^n}) c(\mathcal{T}_{\PP^n}).	\]
Since $c(\mathcal{O}_{\PP^n}) = 1$, this gives:
\begin{equation}
\label{eq:chernproj}
c(\mathcal{T}_{\PP^n}) = (1+ [\omega_{FS}])^{n+1}.
\end{equation}
We used the result of Exercise 4.4.7, $c_1(\mathcal{O}_{\PP^n}(1)) = [\omega_{FS}]$, the class of the Fubini-Study
K\"{a}hler form on $\PP^n$. Recall that $H^*(\PP^n,\C) = \C[\omega_{FS}]/(\omega_{FS}^{n+1})$, with $\deg [\omega_{FS}]
= 2$. Hence $[\omega_{FS}]^{n+1} = 0$ in \ref{eq:chernproj}.

In particular, \ref{eq:chernproj} gives $c_n(\PP^n) = (n+1) [\omega_{FS}^n]$. By the result of Exercise 3.1.4,
$\PP^n$ is normalized to have volume 1 with respect to $\omega_{FS}^n$, so:
\[	\int_{\PP^n} c_n(\PP^n) = n+1.	\]

Now let $p_1, p_2$ be the projections from $\PP^n \times \PP^m$ to the two factors. Since $\mathcal{T}_{\PP^n \times
\PP^m} = p_1^*(\mathcal{T}_{\PP^n}) \oplus p_2^*(\mathcal{T}_{\PP^m})$, the Whitney sum formula, together with invariance
under pullbacks, give:
\begin{equation}
\label{eq:chernprojj}
c(\mathcal{T}_{\PP^n \times\PP^m}) = p_1^*(c(\mathcal{T}_{\PP^n})) \cdot p_2^*(c(\mathcal{T}_{\PP^m}))
= (1 + p_1^* \omega_{FS,\PP^n})^{n+1} (1 + p_2^* \omega_{FS,\PP^m})^{m+1}.
\end{equation}
Again, $(p_1^* \omega_{FS,\PP^n})^{n+1} = (p_2^* \omega_{FS,\PP^m})^{m+1} = 0$ in \ref{eq:chernprojj}. In particular,
$c_{n+m}(\PP^n\times\PP^m) = (n+1)(m+1) p_1^* \omega_{FS,\PP^n}^n \cdot p_2^* \omega_{FS,\PP^m}^m$. Separation of variables
then implies:
\[	\int_{\PP^n\times\PP^m} c_{n+m}(\PP^n\times\PP^m) =(n+1)(m+1) \int_{\PP^n} \omega_{FS,\PP^n}^n
\int_{\PP^m} \omega_{FS,\PP^m}^m = (n+1)(m+1). 	\]
The integral $\int_X c_n(X)$ is equal to the topological Euler characteristic of $X$. This is a consequence of the 
Hirzebruch-Riemann-Roch theorem, or more specifically the special case thereof known as the Gauss-Bonnet formula. 
(See Corollary 5.1.4, (iii).)



\subsection*{Exercise 4.4.5}
\emph{Prove the following explicit formulae for the first three terms of $\ch(E)$ and $\td(E)$ in terms of $c_i(E)$:}
\begin{align*}
\ch &= \rk + c_1 + \frac{c_1^2 - 2c_2}{2} + \frac{c_1^3 - 3c_1 c_2 + 3c_3}{6} + \dots \\
\td &= 1 + \frac{c_1}{2} + \frac{c_1^2 + c_2}{12} + \frac{c_1c_2}{24} + \dots
\end{align*}
\vspace{3mm}


For the first part, recall the definitions $c_i = e_i(\gamma_k)$ and $\ch_i = \frac{1}{i!} p_i(\gamma_k)$, where
$\gamma_k$ are the Chern roots of $E$, and $e_i, p_i$ are elementary symmetric and sum-power symmetric polynomials,
respectively. We use Newton's formula for symmetric polynomials:
\[	ke_k = \sum_{i=1}^k (-1)^{i-1} e_{k-i} p_i .	\]
This implies:
\[	k c_k = \sum_{i=1}^k (-1)^{i-1} i! c_{k-i} \ch_i.	\]
For the first few terms, we obtain explicitly:
\begin{align*}
c_1 = c_0 \ch_1 = c_1 &\Longrightarrow \ch_1 = c_1, \\
2c_2 = c_1 \ch_1 - 2c_0 \ch_2 = c_1^2 - 2\ch_2 &\Longrightarrow \ch_2 = \frac{c_1^2 - 2c_2}{2}, \\
3c_3 = c_2\ch_1 - 2c_1 \ch_2 + 6c_0\ch_3 = c_2c_1 - c_1(c_1^2 - 2c_2) &\Longrightarrow \ch_3 = 
\frac{c_1^3 - 3c_2c_1 + 3c_3}{6}, \\
4c_4 = c_3c_1 - c_2(c_1^2 - 2c_2) + c_1(c_1^3 - 3c_2c_1 + 3c_3) &\Longrightarrow \ch_4 = 
\frac{c_1^4 - 4c_1^2 c_2 + 2c_2^2 + 4c_3c_1 - 4c_4}{24}.
\end{align*}

For the second part, let $r = \rk(E)$ we start from the definition of $\td_i$ as the coefficient of $t^i$ in:
\begin{align*}
\frac{\det(tF_E)}{\det(I - e^{-tF_E})} &= \frac{t^r \prod_{i=1}^r \gamma_i}{\prod_{i=1}^r (1 - e^{-t\gamma_i})} 
= \frac{t^r \prod_{i=1}^r \gamma_i}{\prod_{i=1}^r \sum_{j=1}^{\infty} \frac{(-1)^{j-1}}{j!} t^j \gamma_i^j} \\
&= \frac{1}{\prod_{i=1}^r \sum_{j=0}^{\infty} \frac{(-1)^{j}}{(j+1)!} t^j \gamma_i^j}
= \frac{1}{1 + a_1 t + a_2 t^2 + a_3 t^3 + \dots} .
\end{align*}
We read off the first few coefficients in the denominator:
\begin{align*}
a_1 &= -\frac{1}{2} \sum_{i} \gamma_i, \\
a_2 &= \frac{1}{4} \sum_{i<j} \gamma_i \gamma_j + \frac{1}{6} \sum_{i} \gamma_i^2, \\
a_3 &= - \frac{1}{12} \sum_{i<j} \gamma_i \gamma_j^2 - \frac{1}{12} \sum_{i<j} \gamma_i \gamma_j^2 - 
\frac{1}{24} \sum_i \gamma_i^3.
\end{align*}
Now $\td$ is the inverse of the power series in the denominator, so:
\begin{align*}
\td_1 &= - a_1 = \frac{1}{2} \sum_i \gamma_i = \frac{c_1}{2}, \\
\td_2 &= a_1^2 - a_2 = \frac{3 \left(\sum_i \gamma_i\right)^2 - 3 \sum_{i<j} \gamma_i \gamma_j + 2 \sum_i \gamma_i^2}{12}
= \frac{3c_1^2 - 3c_2 - 2(c_1^2 - 2c_2)}{12} = \frac{c_1^2 + c_2}{12}, \\
\td_3 &= - a_1^3 + 2a_1a_2 - a_3 = \frac{1}{8} c_1^3 - \frac{1}{4} c_1c_2 - \frac{1}{3} c_1 \ch_2 + \frac{1}{4} \ch_3
+ \frac{1}{36}(c_1^3 - 6 \ch_3 - 6c_3) = \frac{c_1c_2}{24}.
\end{align*}



\subsection*{Exercise 4.4.6}
\emph{Let $E$ be a vector bundle and $L$ a line bundle. Show:}
\[	c_i(E \otimes L) = \sum_{j=0}^i \binom{\rk(E) - j}{i-j} c_j(E) c_1(L)^{i-j} .	\]
\vspace{3mm}

Recall that $F_{E\otimes L} = F_E \otimes I_1 + I_{\rk(E)} \otimes F_L = F_E + F_L I_{\rk(E)}$. $c_i(E \otimes L)$
is the degree $2i$ part of:
\[	\det(I + F_{E\otimes L}) = \prod_{i=1}^{\rk(E)} (1 + \gamma_i + c_1(L)),	\]
where $\gamma_i$ are the Chern roots of $E$. It follows that $c_i(E\otimes L) = e_i(\gamma_i + c_1(L))$, where $e_i$
denotes the $i^{\text{th}}$ elementary symmetric polynomial. Each term in $e_i(\gamma_i + c_1(L))$ is of the form:
\begin{equation}
\label{eq:prodexp}
	\prod_{k=1}^i (\gamma_k + c_1(L)) = \sum_{j=1}^i \gamma_{k_1} \dots \gamma_{k_j} c_1(L)^{j-i} .
\end{equation}
Summing over all terms of $e_i(\gamma_i + c_1(L))$, we obtain:
\[	c_i(E\otimes L) = \sum_{j=1}^i a_j c_j(E) c_1(L)^{j-i},	\]
where the numerical coefficient $a_j$ counts the number of times each monomial $\gamma_{k_1} \dots \gamma_{k_j}$
appears. This is equivalent to counting the ways in which one can choose $j-i$ factors of $c_1(L)$ in the expansion of
the product in \ref{eq:prodexp}, out of a total of $\rk(E) - j$ factors which are not chosen to be $\gamma_k$.
Hence $a_j = \binom{\rk(E) - j}{i-j}$.





\subsection*{Exercise 4.4.7 in \cite{Huy}}
\emph{Show that on $\PP^n$ one has $c_1(\mathcal{O}(1)) = [\omega_{FS}] \in H^2(\PP^n,\R)$. Consider first the
case of $\PP^1$ and then the restriction of $\mathcal{O}(1)$ and the Fubini-Study metric to $\PP^1$ under
a linear embedding $\PP^1 \subset \PP^n$.}
\vspace{3mm}

The result follows immediately from Example 4.3.12, according to which the curvature of the Chern connection on
$\mathcal{O}_{\PP^n}(1)$ satisfies $F = \frac{2\pi}{i} \omega_{FS}$. (This result, which is local, is proved using
a simple computation in coordinates.)

Alternatively, we can take $c_1(\mathcal{O}_{\PP^1}(1)) = [\omega_{FS,\PP^1}]$ as hypothesis, and prove that the
result holds for all $n$. Embed $\PP^1$ in $\PP^n$ 
via the map $f$:
\[	[x:y] \mapsto [x:y:0:\dots:0].	\]
We have $f^*(\mathcal{O}_{\PP^n}(1)) = \mathcal{O}_{\PP^1}(1)$, so $f^* c(\mathcal{O}_{\PP^n}(1)) = c
(\mathcal{O}_{\PP^1}(1))$. Moreover, $f^*(\omega_{FS,\PP^n}) = \omega_{FS,\PP^1}$, which
is obtained by setting $w_2 = \dots = w_n = 0$ in the coordinate expression of $\omega_{FS,\PP^n}$:
\[	\omega_{FS,\PP^n} = \bar \p \p \log(1 + \sum_{i=1}^n |w_i|^2).	\]

In the resulting 2 equations:
\begin{align*}
f^* &c_1(\mathcal{O}_{\PP^n}(1)) = c_1(\mathcal{O}_{\PP^1}(1)) \\
f^* &[\omega_{FS,\PP^n}] = [\omega_{FS,\PP^1}]
\end{align*}
the morphism $f^* : H^2(\PP^n,\C) \to H^2(\PP^1,\C)$ maps elements of $H^2(\PP^n,\C) \cong \Z$ to a generator of
$H^2(\PP^1,\C) \cong \Z$. Therefore $c_1(\mathcal{O}_{\PP^n}(1))$ and $[\omega_{FS,\PP^n}]$ are themselves
generators of $H^2(\PP^n,\C) \cong \Z$, which implies $c_1(\mathcal{O}_{\PP^n}(1)) = \pm [\omega_{FS,\PP^n}]$.
We also know that $f^* c_1(\mathcal{O}_{\PP^n}(1)) = f^* [\omega_{FS,\PP^n}]$, which fixes the sign.


\subsection*{Exercise 4.4.8}
\emph{Prove that a polynomial $P$ of degree $k$ on the space of $r\times r$ matrices is invariant if and only if}
\[ \sum P(A_1, \dots, A_{i-1}, [A,A_i] , A_{i+1}, \dots, A_k) = 0 \]
{for all matrices $A_1, \dots, A_k, A$ (cf. Lemma 4.4.2).}
\vspace{3mm}

We use the fact that any $C \in \GL(r,\C)$ can be written as $C = e^A$ for some $A \in M_r(\C)$. This is proved by
considering the Jordan form of $C$, writing each block as $\lambda I + N$, with $N$ nilpotent, and then defining the
logarithm of the block as: 
\[	\log (\lambda) I + \sum_{i=1}^{\infty} \frac{\lambda^{-i}}{i} N^i.  \]
Note that the sum is actually finite, due to the nilpotency of $N$. Moreover, since $C$ has finitely many Jordan blocks,
and all $\lambda \neq 0$, we can choose a branch of the complex logarithm such that $\log (\lambda)$ is defined for all
$\lambda$.

With this in mind, note that:
\[	\frac{d}{dt}\left. P(e^{tA}A_1e^{-tA}, \dots, e^{tA}A_k e^{-tA})\right|_{t=0} = 
\sum_{j=1}^k P(A_1, \dots, A_{j-1}, [A,A_j] , A_{j+1}, \dots, A_k).	\]
If $P$ is invariant, then the LHS is the derivative of a constant, so the RHS is 0. Conversely, if the RHS is 0,
then the quantity under the derivative in the LHS must be constant. The value of the constant is determined by evaluating
at $t=0$, which gives $P(A_1, \dots, A_k)$.



\subsection*{Exercise 4.4.9 in \cite{Huy}}
\emph{Show that $c_1(\End(E)) = 0$ on the form level and compute $c_2(\End(E))$ in
terms of $c_i(E)$, $i = 1,2$. Compute $(4c_2 — c_1^2)(L \oplus L)$ for a line bundle $L$. Show that
$c_{2k+1}(E) = 0$, if $E\cong E^*$.}
\vspace{3mm}

Recall that a connection $\nabla$ on $E$ induces the following connection on $\End(E)$, which we also denote by $\nabla$.
\[	\nabla(f) (s_1) = \nabla(f(s_1)) - f(\nabla(s_1)).	\]
Applying $\nabla$ again we obtain:
\begin{align*}
(\nabla^2 f) (s_1) &= \nabla(\nabla(f)) (s_1) = \nabla \big( \nabla(f) (s_1) \big) + \nabla(f) \big(\nabla(s_1)\big) \\
&= \nabla^2\big(f(s_1)\big) - \nabla\big(f(\nabla(s_1))\big) + \nabla(f) \big(\nabla(s_1)\big) \\
&= \nabla^2\big(f(s_1)\big) - \nabla^2\big(f(s_1)\big) - f\big(\nabla^2(s_1)\big) +  \nabla(f) \big(\nabla(s_1)\big) \\
&= \nabla^2\big(f(s_1)\big) - f\big(\nabla^2(s_1)\big).
\end{align*}
The sign is flipped whenever $\nabla$ passes a 1-form, with which it anticommutes. Therefore $F_{\End(E)} = F_E \otimes 1 - 
1 \otimes F_E^T$. When computing the Chern character, traces are applied to powers $F$, and traces do not distinguish
between $F_E$ and $F_E^T$. Hence we can replace $F_E^T$ by $F_E$ in what follows. Let $r=\rk(E)$, then:
\[	\ch(\End(E)) = \tr(e^{\frac{i}{2\pi}F_E}) \tr(e^{-\frac{i}{2\pi}F_E}) = \big(r+ \ch_1(E) +	\ch_2(E) + \dots \big)
\big(r - \ch_1(E) +	\ch_2(E) + \dots \big). \]
In particular:
\begin{align*}
c_1(\End(E)) = \ch_1(\End(E)) &= -r\ch_1(E) + r \ch_1(E) = 0 ,\\
- c_2(\End(E)) = \ch_2(\End(E)) &= 2r \ch_2(E) - \ch_1^2(E) = (r-1) c_1^2(E) - 2r c_2(E) , \\
\end{align*}
therefore:
\begin{equation}
\label{eq:c2end}
c_1(\End(E)) = 0, \hspace{10mm} c_2(\End(E)) = 2r c_2(E) - (r-1)c_1^2(E).
\end{equation}

If $L$ is a line bundle, then:
\[	c(L\oplus L) = (1 + c_1(L))^2 = 1 + 2c_1(L) + c_1^2(L).	\]
Therefore:
\[	(4c_2 - c_1^2)(L\oplus L) = 4c_1^2(L) - 4c_1^2(L) = 0.	\]
Together with equation \ref{eq:c2end}, this means that $c_2(\End(L\oplus L)) = 0$.

Finally, if $E \cong E^*$, then $F = -F^T$, due to Proposition 4.3.7, (iii). However, $F$ is also skew-adjoint,
i.e. $\bar F = - F^T$, so in this case it is real, skew-symmetric. 
It follows that its eigenvalues (in some extension ring $R$ of
$\Lambda^* E$) are purely imaginary. Then $c_{2k+1}(E)$ are also purely imaginary. However, Chern classes are
real: see Remark 4.4.9 (i). Thus $c_{2k+1}(E) = 0$. Note that this agrees with equation \ref{eq:c2end}, as it should,
because $\End(E) \cong E \otimes E^* \cong \End(E)^*$.




\subsection*{Exercise 4.4.10 in \cite{Huy}}
\emph{Let $L$ be a holomorphic line bundle on a compact Kahler manifold
$X$. Show that for any closed real (1,1)-form a with $[a] = c_1(L)$ there exists an
hermitian structure on $L$ such that the curvature of the Chern connection $\nabla$ on $L$
satisfies $(i/2\pi)F_{\nabla} = a$. (Hint: Fix an hermitian structure $h_0$ on $L$. Then any
other is of the form $e^f h_0$. Compute the change of the curvature. We will give the
complete argument in Remark 4.B.5.)}
\vspace{3mm}

Let $\alpha_0 = \frac{i}{2\pi} F_{h_0}$ and $\alpha = \frac{i}{2\pi} F_{h}$, where $h = e^f h$. Then:
\[	\alpha - \alpha_0 = \frac{i}{2\pi} \left( \bar \p \p (f + \log h) - \bar \p \p (\log h )\right) = \frac{i}{2\pi} 
\bar \p \p f.	\]
Hence the question reduces to finding $f \in C^{\infty}(X)$ for any prescribed value of $\bar \p \p f \in \mathcal{A}^{1,1}
(X)$. This is always possible, because of the $\p \bar \p$ lemma. (See Corollary 3.2.10 in \cite{Huy}.)


\subsection*{Exercise 4.4.11 in \cite{Huy}}
\emph{Let $X$ be a compact K\"{a}hler manifold. Show that via the natural inclusion
$H^k(X, \Omega^k_X) \subset H^{2k}(X,\C)$ one has:}
\[	\ch_k(E) = \frac{1}{k!} \left(\frac{i}{2\pi}\right)^k \tr \left(A(E)^{\otimes k} \right) .	\]
\emph{Here, $A(E)^{\otimes k}$ is obtained as the image of $A(E) \otimes \dots \otimes A(E)$ under the natural map
$H^1(X, \Omega_X \otimes \End(E)) \times \dots \times H^1(X, \Omega_X \otimes \End(E)) \to H^k(X, \Omega^k \End(E))$ which
is induced by composition in $\End(E)$ and exterior product in $\Omega_X$.}
\vspace{3mm}

In Proposition 4.3.10 it is proved that $[F] = A(E) \in H^1(X,\Omega_X \otimes \End(E))$, where $F$ is the curvature of the
Chern connection on $E$. The proof uses the equivalence of Dolbeault and $\check{C}$ech cohomology, which preserves
the ring structure of the two cohomologies. In particular, $[F^k] = A(E)^{\otimes k}$ as elements of
$H^k(X, \Omega^k \End(E))$. Then the classes of $\tr F^k$ and $\tr A(E)^{\otimes k}$ are equal as elements of
$H^k(X,\Omega^k)$, and also as elements of $H^{2k}(X,\C)$, using the inclusion $H^k(X, \Omega^k_X) \subset H^{2k}(X,\C)$.
This gives the result.


\subsection*{Exercise 4.4.12 in \cite{Huy}}
\emph{Let $X$ be a compact Kahler manifold and let $E$ be a holomorphic vector
bundle admitting a holomorphic connection $D : E \to \Omega_X \otimes E$. Show that $c_k(E) = 0$
for all $k > 0$.}
\vspace{3mm}

Due to Proposition 4.2.19, the existence of $D$ implies that $A(E) = 0$. Using Exercise 4.4.11, $\ch_k(E) = 0$ for all
$k>1$. Using inductively the expressions for $\ch_k(E)$ in terms of $c_1(E), \dots, c_k(E)$, we deduce that $c_k(E) = 0$
for all $k>1$.




\section{Huybrechts, Chapter 5}

\subsection*{Exercise 5.1.1 in \cite{Huy}}
\emph{Let $X$ be a K3 surface (cf. Exercise 2.5.5). Show that $b_2(X) = 22$. Prove that the Picard number $\rho(X)$
is bounded by 20.}
\vspace{3mm}

Since $\omega_X = \Omega_X^{\wedge 2} = \mathcal{O}_X$, $h^{2,0} = 1$. Then $h^{0,2} = 1$ also, and by hypothesis we have:
\[	\chi(X,\mathcal{O}_X) = h^{0,0} - h^{0,1} + h^{0,2} = 2.	\]
But this is the same as the Hirzebruch $\chi_0$ genus, so:
\begin{equation}
\label{eq:k3char}
	2 = \int_X \td(X) = \int_X \frac{c_1^2(X) + c_2(X)}{12} .
\end{equation}
We compare now $\chi(X,\Omega_X)$ and $\chi(X,T_X)$. Due to Serre duality and the fact that $\omega_X \cong \mathcal{O}_X$,
$\chi(X,\Omega_X) = \chi(X,T_X)$. However, $c_k(\Omega_X) = (-1)^k c_k(T_X)$, so these can be computed from Hirzebruch-Riemann-Roch:
\begin{align*}
\chi(\Omega_X) &= \int_X \ch(\Omega_X) \td(X) = \int_X 2 \frac{c_1^2(X) + c_2(X)}{12} + \int_X -c_1(X) \frac{c_1(X)}{2}
+ \int_X \frac{c_1^2(X) - 2c_2(X)}{2}, \\
\chi(T_X) &= \int_X \ch(T_X) \td(X) = \int_X 2 \frac{c_1^2(X) + c_2(X)}{12} + \int_X c_1(X) \frac{c_1(X)}{2}
+ \int_X \frac{c_1^2(X) - 2c_2(X)}{2}.
\end{align*}
The equality of these quantities gives $\int_X c_1^2(X) = 0$. Using this information in \ref{eq:k3char}, we obtain
$\int_X c_2(X) = 24$. But this is just the Hirzebruch -1 genus, i.e. the topological Euler characteristic of $X$.
Hence:
\[	24 = b_0(X) + b_2(X) + b_4(X) = 1 + b_2(X) + 1 \Longrightarrow b_2(X) = 22.	\]
Together with the fact that $h^{2,0}(X) = h^{0,2}(X) = 1$, this implies $h^{1,1}(X) = 20$. In fact, 
we have obtained the full Hodge diamond of a K3 surface.
\[
\begin{tikzcd}
\; & & 1 & & \\
 &0 &  &0 & \\
1& & 20 & &1 \\
 &0 &  &0 & \\
 & & 1 & &
\end{tikzcd}
\]
Recall that $\rho(X) = \rk(H^{1,1}(X,\Z))$, due to the Lefschetz theorem on (1,1) classes. Therefore
$\rho(X) \leq h^{1,1}(X) = 20$.


\subsection*{Exercise 5.1.2 in \cite{Huy}}
\todo{do this}


\subsection*{Exercise 5.1.3 in \cite{Huy}}
\emph{The Hilbert polynomial of a polarized manifold $(X,L)$, i.e. $L$ is an ample line bundle on $X$, is defined as the
function:}
\[	\Z \to \Z, \hspace{3mm} m \mapsto P_(X,L)(m):= \chi(X,L^{\otimes m}).	\]
\emph{Show that $P_(X,L)$ is indeed a polynomial. Determine its degree and leading coefficient.}
\vspace{3mm}

Recall that, for a line bundle $L$,	$\ch(L) = e^{c_1(L)}$. Since the Chern character is a homomorphism of semirings,
$\ch(L^{\otimes m}) = (e^{c_1(L)})^m = e^{mc_1(L)}$. We plug this into Hirzebruch-Riemann-Roch and expand the exponential:
\begin{equation*}
\chi(X,L^{\otimes m}) = \int_X \ch(L^{\otimes m}) \td(X) = \int_X \left(1 + m c_1(L) + \frac{m^2c_1^2(L)}{2}
+ \dots + \frac{m^nc_1^n(L)}{n!} \right) \td(X)  .
\end{equation*}
This is a polynomial in $m$ of degree $n = \dim X$. We can read off the coefficients $a_k$ as:
\[		a_k = \int_X \frac{c_1(L)^k}{k!} \td_{n-k}(X) .	\]
In particular, the leading coefficient is:
\[	a_n = \int_X \frac{c_1(L)^n}{n!} .	\]




\subsection*{Exercise 5.1.4 in \cite{Huy}}
\emph{Compute the Hilbert polynomial of a hypersurface $Y\subset \PP^n$ of degree $k$.}
\vspace{3mm}

From Hirzebruch-Riemann-Roch, we have:
\[	\chi(Y,L^m) = \int_Y \ch(L^m) \td(Y).	\]
The natural polarization of $Y$ is given by $L = \iota^* \mathcal{O}_{\PP^n}(1)$. Let $\alpha = \iota^*[\omega_{FS}] \in
H^{1,1}(Y,\Z)$. It follows that:
\[	\ch(L^m) = e^{m c_1(L)} = e^{m\alpha}.	\]
To compute $\td(Y)$, we use the multiplicative property of $\td$ over short exact sequences, applied to the normal bundle
sequence:
\[	0 \to TY \to T\PP^n|_Y \to \mathcal{O}(k)|_Y \to 0.	\]
Then $\td(Y) \iota^*\td(\mathcal{O}(k)) = \iota^* \td(\PP^n)$. From the Euler sequence, we have:
\[	\td(\PP^n) = \td(\mathcal{O}(1))^{n+1} = \left( \frac{[\omega_{FS}]}{1 - e^{-[\omega_{FS}]}} \right)^{n+1}.	\]
Moreover, using the $k$-uple Veronese embedding $f :\PP^n \to \PP^N$:
\[	\td(\mathcal{O}(k)) = f^* \td(\mathcal{O}(1)) = \frac{k [\omega_{FS}]}{1 - e^{-k[\omega_{FS}]}} . \]
Putting everything together, we obtain:
\[	\chi(Y,L^m) = \int_Y e^{m \alpha} \frac{1 - e^{-k\alpha}}{k \alpha} 
\left( \frac{\alpha}{1 - e^{-\alpha}} \right)^{n+1} .	\]
I'm not sure how to isolate the degree $n-1$ part of the integrand. \todo{figure this out}


\subsection*{Exercise 5.1.5 in \cite{Huy}}
\emph{Let $L$ be a line bundle on a compact connected curve $C$ with $\deg(L)>g-1$. Show that $L$ admits non-trivial global
holomorphic sections.}
\vspace{3mm}

We use Riemann-Roch for line bundles on curves:
\[	h^0(X,L) - h^1(X,L) = \deg(L) - g + 1 > 0.	\]
The two quantities on the LHS are positive integers, so in particular $h^0(X,L) > 0$.



\subsection*{Exercise 5.1.6 in \cite{Huy}}
\emph{Let $L$ be a line bundle on a compact connected curve $C$ with $\deg(L)>2g-1$. Show that $L$ is globally generated.}
\vspace{3mm}

Let $d = \deg L$. We apply Riemann-Roch for $L$:
\[	h^0(X,L) = h^1(X,L) + d - g + 1 \geq d - g + 1. 	\]
In particular, since $d>2g-1$, $h^0(X,L) > g \geq 0$, so $L$ admits a global section. Now $\Gamma(X,L)$ fails to generate
the stalk $L_x$ at $x \in C$ if and only if $H^0(X,L) = H^0(X,L\otimes \mathcal{O}(-x))$. Applying Riemann-Roch
again:
\[	h^0(X,L\otimes \mathcal{O}(-x)) = h^0(X,K_X \otimes L^{-1} \otimes \mathcal{O}(x)) + (d-1) - g + 1.	\]
But $\deg (K_X \otimes L^{-1} \otimes \mathcal{O}(x)) < 0$, so $h^0(X,K_X \otimes L^{-1} \otimes \mathcal{O}(x)) = 0$.
Hence:
\[	h^0(X,L\otimes \mathcal{O}(-x)) = d - g.	\]
Comparing with the bound on $h^0(X,L)$, we see that the two cannot be equal. Therefore $L$ is generated by global sections.



\subsection*{Exercise 5.1.7 in \cite{Huy}}
\emph{Let $L$ be a line bundle on a compact surface $X$ with $\int_X c_1^2(L) > 0$. Show that for $m>>0$ either
$L^{\otimes m}$ or $L^{\otimes (-m)}$ admits non-trivial global holomorphic sections.}
\vspace{3mm}

We use Hirzebruch-Riemann-Roch:
\begin{align*}
\chi(X,L^{\otimes m}) &= \chi(X,\mathcal{O}_X) + \int_X \frac{c_1(L^{\otimes m}) [ c_1(L^{\otimes m}) + c_1(X) ]}{2} \\
&= \chi(X,\mathcal{O}_X) + \int_X \frac{mc_1(L)[mc_1(L) + c_1(X)]}{2} \\
&= m^2 \int_X \frac{c_1^2(L)}{2} + m \int_X \frac{c_1(L)c_1(X)}{2} + \chi(X,\mathcal{O}_X) .
\end{align*}
By hypothesis the leading coefficient is positive. Therefore, when $|m|>>0$, $\chi(X,L^{\otimes m}) > 0$. In other words:
\[	h^0(X,L^{\otimes m}) - h^1(X,L^{\otimes m}) + h^2(X,L^{\otimes m}) > 0.	\]
It suffices to show that, for either $m>>0$ or $m<<0$, $h^2(X,L^{\otimes m}) = 0$. Using Serre duality, this is equivalent
to $h^0(X,\omega_X \otimes L^{-\otimes m}) = 0$. \todo{finish this}\footnote{If $L$ is ample, then for $m$ large enough
$L^{\otimes m}$ is very ample. Then $L^{-\otimes m} = \mathcal{O}_X(-D)$ for some divisor $D$, and any section of this
line bundle would have zeros. However, $H^0(\mathcal{O}_X(-D)) \subset H^0(\mathcal{O}_X)$, which is a field.}



\subsection*{Exercise 5.1.8 in \cite{Huy}}
\emph{Let $X$ be a compact surface such that $c_1(X) \in 2H^2(X,\Z)$. Use Hirzebruch-Riemann-Roch to show that
$\int_X c_1^2(L)$ is even for any line bundle $L$ on $X$.}
\vspace{3mm}

Using Hirzebruch-Riemann-Roch for surfaces, we have that:
\[	\chi(X,L) - \chi(X,\mathcal{O}_X) = \int_X \frac{c_1^2(L)}{2} + \int_X \frac{c_1(L)c_1(X)}{2} .	\]
Now recall that $c_1(L) \in H^2(X,\Z)$ for every line bundle $L$: according to Proposition 4.4.12, it agrees up to a sign with
the image of $L \in \Pic(X) \cong H^1(X,\mathcal{O}^{\times})$ under the boundary map of the exponential sequence.
Then the quantity:
\[	\int_X \frac{c_1(L)c_1(X)}{2},	\]
which is equal to the evaluation of $c_1(L) \smile c_1(X) / 2 \in H^4(X,\Z)$ on the fundamental class $[X]\in H_4(X,\Z)$,
is an integer. But $\chi(X,L) - \chi(X,\mathcal{O}_X) \in \Z$ also. Therefore $\int_X c_1^2(L) \in 2\Z$.



\subsection*{Exercise 5.1.9 in \cite{Huy}}
\emph{Let $X$ and $Y$ be compact complex manifolds and let $f:X \to Y$ be a smooth finite morphism of degree $d$.
In other words, $f:X \to Y$ is smooth surjective with $\dim(X) = \dim(Y)$ and every fiber $f^{-1}(y)$ contains
$d$ points.\footnote{Counted with multiplicity, I assume.}}

\emph{Show that $\td(X) = f^*(\td(Y))$ and deduce $\chi(X,\mathcal{O}_X) = d \chi(Y,\mathcal{O}_Y)$. In particular, if
$X$ and $Y$ are K3 surfaces, then $d=1$.}
\vspace{3mm}

Since $f$ is smooth and finite, it induces an isomorphism on tangent spaces at each point. Therefore $TX = f^*(TY)$,
so $c(X) = f^*(c(Y))$, and then the relation follows formally for $\td$ as well.  Then:
\[	\chi(X,\mathcal{O}_X) = \int_X \td_n(X) = \int_X f^*(\td_n(Y)) = d \int_Y \td_n(Y) = d \chi(Y,\mathcal{O}_Y).	\]
In particular, if both $X$ and $Y$ are K3 surfaces, then $\chi(X,\mathcal{O}_X) = \chi(Y,\mathcal{O}_Y) = 2$, due to
the condition $h^1(X,\mathcal{O}_X) = 1$. The only possibility is $d=1$.



\subsection*{Exercise 5.2.1 in \cite{Huy}}
\emph{Let $(E, h)$ be an hermitian holomorphic vector bundle on a compact K\"{a}hler
manifold $X$. Suppose that the curvature $F_{\nabla}$ of the Chern connection is trivial, i.e.
the Chern connection is flat. Prove that the Lefschetz operator $\Lambda$ preserves the
harmonicity of forms and thus defines a map $\Lambda : \mathcal{H}^{p,q}(X,E) \to \mathcal{H}^{p-1,q-1}(X,E)$.
Deduce from this the existence of a Lefschetz decomposition on $H^{*'*}(X,E)$.}
\vspace{3mm}


\subsection*{Exercise 5.2.2 in \cite{Huy}}
\emph{Let $C$ be an elliptic curve. Show that $H^1(C,\mathcal{O}_C) = \C$ and use this to
construct a non-splitting extension $0\to \mathcal{O} \to E \to \mathcal{O} \to 0$. Prove that $E$ cannot be
written as a direct sum of two holomorphic line bundles.}

\emph{(There is an algebraic argument using $H^1(C,\mathcal{O}_C) = \Ext^1(\mathcal{O}_C, \mathcal{O}_C)$, but one
might as well try to construct a new $\bar \p$-operator on the trivial bundle of rank two
by means of a representative of a non-trivial class in $H^1(C, \mathcal{O}_C) = H^{0,1}(C)$.)}
\vspace{3mm}

The algebraic argument is easy. $C \cong V/\Gamma$, where $\dim_{\C} V = 1$ and $\Gamma$ is a lattice. Then 
$H^1(C,\mathcal{O}_C) \cong \overline{V^*} \simeq \C$. Now extensions:
\[ 0\to \mathcal{O}_C \to E \to \mathcal{O}_C \to 0	\]
are parametrized by
\[	\Ext^1(\mathcal{O}_C,\mathcal{O}_C) \cong H^1(\mathcal{O}_C^{-1} \otimes \mathcal{O}_C) \cong H^1(\mathcal{O}_C) \simeq \C.	\]
$0 \in \C$ corresponds to the trivial extension, but taking $1 \in \C$ gives a non-split extension.


\subsection*{Exercise 5.2.3 in \cite{Huy}}
\emph{Show that on $\PP^2$ there exists a rank two vector bundle which is not isomorphic
to the direct sum of holomorphic line bundles.}
\vspace{3mm}

We show that the tangent bundle of $\PP^2$ is indecomposable. In fact, we use a characteristic class argument, which
proves that $T\PP^2$ does not even split as a complex vector bundle. Recall the Euler sequence:
\[	0 \to \mathcal{O}_{\PP^2} \to \mathcal{O}_{\PP^2}(1)^{\oplus 3} \to T\PP^2 \to 0.	\]
Writing $c$ for the total Chern class, this implies $c(T\PP^2) = c(\mathcal{O}_{\PP^2}(1))^3$. Let
$\alpha = c_1(\mathcal{O}_{\PP^2}(1))$, and recall that the integral cohomology ring of $\PP^2$ is $\Z[\alpha]/(\alpha^3)$.
Then:
\[	c(T\PP^2) = 1 + 3\alpha + 3\alpha^2.	\]
If $T\PP^2$ were decomposable as $L_1 \oplus L_2$ topologically, the Whitney sum formula would imply $c(T\PP^2) = c(L_1) c(L_2)$.
But $L_i$ are line bundles, so $c(L_i) = 1 + n_i \alpha$, with $n_i \in \Z$. Then:
\[	1 + 3 \alpha + 3 \alpha^2 = 1 + (n_1 + n_2) \alpha + n_1n_2 \alpha^2.	\]
This equation has no solutions in $\Z$.



\subsection*{Exercise 5.2.4 in \cite{Huy}}[The degree-genus formula.]
\emph{Let $C \subset \PP^2$ be a smooth curve defined by
a homogeneous polynomial of degree $d$. Show that the genus $g(C) = \dim H^0(C, K_C)$
is given by the formula}
\[	g_C = \frac{1}{2}(d-1)(d-2).	\]
\emph{Use this to show that there are curves which are not plane, i.e. not isomorphic
to a smooth curve in $\PP^2$. Prove that for a smooth curve $C \subset X$ in a K3 surface $X$
one has $g(C) = ([C]^2 + 2)/2$.}
\vspace{3mm}

From the adjunction formula, plus the description of the normal bundle of a hypersurface given in Exercise 2.3.3,
we obtain:
\[	\omega_C \cong \omega_{\PP^2}|_{C} \otimes \mathcal{O}_{\PP^2}(d)|_{C} \cong \mathcal{O}_{\PP^2}(d-3)|_C.	\]
The degree of $\omega_C$ is $2g-2$, due to Riemann-Roch for curves. By using Bezout and the formula above:
\[	2g-2 = \deg \omega_C = d(d-3),	\]
which gives the degree-genus formula. This shows that not all genera can be realized by plane curves. In particular, no
value of $d$ gives $g=2$.\footnote{Technically, one needs to prove the existence of genus 2 curves, but we'll take it
as known that these can be realized as degree 3 curves in $\PP^3$.}

Applying the adjunction formula for a curve $C$ in a K3 surface $X$, we have:
\[	\omega_C \cong \mathcal{O}_X|_C \otimes \mathcal{O}_X(C)|_C = \mathcal{O}_X(C)|_C.	\]
It follows that the genus is related to the self-intersection of $C$ as:
\[	2g-2 = C.C	\]



\subsection*{Exercise 5.2.5 in \cite{Huy}}
\emph{Show that hypersurfaces in $\PP^n$ with $n \geq 3$ do not admit non-trivial holomorphic
one-forms. In particular, the Albanese of any such hypersurface is trivial.}
\vspace{3mm}

We apply Weak Lefschetz to obtain an isomorphism $H^1(Y,\C) \cong H^1(\PP^n,\C) = 0$. In particular,
$H^0(Y,\Omega_Y) = H^{1,1}(Y)$ is a direct summand of $H^1(Y,\C)$, so it is 0. The Albanese is defined as a
quotient of $H^0(Y,\Omega_Y)^*$, so it too must be trivial.




\subsection*{Exercise 5.2.6 in \cite{Huy}}
\emph{Which complex tori could possibly be realized as complete intersections in $\PP^n$?}
\vspace{3mm}

If $\dim X = 1$, the complex torus $X$ is an elliptic curve, and these are projective. (Given by a cubic in
$\PP^2$, or the intersection of two quadrics in $\PP^3$; see Exercise 2.3.3.)

If $\dim X = m >1$, realizing $X$ as a complete intersection in $\PP^n$ would give a sequence
$X = X_m \subset X_{m+1} \subset \dots \subset X_n = \PP^n$, where each $X_i$ is a hypersurface in $X_{i+1}$. All
$X_i$ defined this way are projective, so in particular K\"{a}hler, and we can apply Weak Lefschetz to obtain
isomorphisms $H^1(X_{i},\C) \to H^1(X_{i-1},\C)$ for $m+1\leq i \leq n$. (Note that $m+1\geq 3$, so we are in the range
where Weak Lefschetz gives isomorphisms on $H^1$.) Hence $H^1(X,\C) \cong H^1(\PP^n,\C) = 0$, by composing the isomorphisms
in this chain.

However, a complex torus $\C^m/\Gamma$ cannot have vanishing first cohomology, because $H^1(X,\C) = \Gamma^* \otimes \C$.



\subsection*{Exercise 5.2.7 in \cite{Huy}}
\emph{Let $L$ be an ample line bundle on a K3 surface $X$. Show that $h^0(X, L) = 2 + (1/2) \int_X c_1(L)^2$. 
Study ample line bundles on complex tori.}
\vspace{3mm}

If $L$ is ample, then there exists a natural number $m$ such that the quantity:
\[	m c_1(L) = c_1(L^{\otimes m}) = f^*(c_1(\mathcal{O}_{\PP^n}(1))) = f^*([\omega_{FS}])	\]
can be represented by a K\"{a}hler form $\omega$. In particular, this means $mc_1(L)$ is a positive form, so
$c_1(L)$ must also be. Hence $L$ is positive, and we can apply Kodaira vanishing:
\[	H^q(X,L\otimes \omega_X) = 0, \text{ for } q>0.	\]
If $X$ is a K3, then $\omega_X \cong \mathcal{O}_X$, so we actually obtain $H^q(X,L) = 0$ for $q>0$. Then
$\chi(X,L) = h^0(X,L)$. As part of the solution to Exercise 5.1.1, we proved that $c_1(X) = 0$ for a K3,
so Hirzebruch-Riemann-Roch reduces to:
\[	\chi(X,L) = \int_X \frac{c_1(L)^2}{2} + \chi(X,\mathcal{O}_X). 	\]
Recall that K3's are required to satisfy $h^1(X,\mathcal{O}_X) = 0$, and that $h^2(X,\mathcal{O}_X) = h^{0,2}(X)
= h^{2,0}(X) = h^0(X,\omega_X) = 1$. Then $\chi(X,\mathcal{O}_X) = 2$. Putting everything together gives:
\[	h^0(X,L) = \int_X \frac{c_1(L)^2}{2} + 2.	\]

Similarly, if $X$ is a complex torus, then $T_X$ is trivial, so $\omega_X \cong \mathcal{O}_X$. Kodaira vanishing
then gives $H^q(X,L) = 0$ for $q>0$. Because $T_X$ is trivial, $\td(X) = 1$, so Hirzebruch-Riemann-Roch gives:
\[	h^0(X,L) = \chi(X,L) = \int_X \frac{c_1(L)^n}{n!} .	\]
See Exercise 5.1.2 for how to compute the RHS, in terms of a matrix representation for $c_1(L)$.




\subsection*{Exercise 5.2.8 in \cite{Huy}}
\emph{Use Serre duality to give a direct algebraic proof of the Kodaira vanishing theorem for curves.}
\vspace{3mm}

If $\dim X = 1$, then $\Omega_X = \omega_X$, and $H^q = 0$ for all vector bundles and $q>1$.
\footnote{It's isomorphic to $H^{1-q}$ of another vector bundle, using Serre duality.} Therefore $p+q>1$ can only
be achieved with $p=q=1$, so Kodaira vanishing for curves is the statement that $H^1(X,\omega_X \otimes \mathscr{L}) =0$
for any positive line bundle $\mathscr{L}$. By Serre duality, this is dual to $H^0(X,\mathscr{L}^{-1}) = 0$. For curves,
$c_1(\mathscr{L}) \in H^{1,1}(X,\Z) \cong \Z$, so positivity is equivalent to $\deg \mathscr{L} >0$. Then
$\deg \mathscr{L}^{-1} < 0$, and $\mathscr{L}^{-1}$ has no global sections.


\subsection*{Exercise 5.2.9 in \cite{Huy}}
\emph{Prove that $H^q(\PP^n, \Omega^p(m)) = 0$ for $p + q > n$, $m > 0$ and for $p + q < n$,
$m < 0$.}
\vspace{3mm}

The first part is immediate from Kodaira vanishing: the line bundles $\mathcal{O}_{\PP^n}(m)$ for $m>0$ are positive,
because $c_1(\mathcal{O}_{\PP^n}(m)) = m \omega_FS$. For the second part, replace $m$ by $-m$ and apply Serre duality:
\[	H^q(\PP^n, \Omega^p(-m)) \cong H^{n-q}(\PP^n, \Omega^n \otimes \Omega^{p*} \otimes \mathcal{O}(m)).	\]
\todo{finish this}


\subsection*{Exercise 5.2.10 in \cite{Huy}}
\emph{Let $Y$ be a hypersurface of a compact complex manifold $X$ with $\mathcal{O}(Y)$
positive. Suppose that $H^2(X,\Z)$ and $H^2(Y, \Z)$ are torsion free. Prove that the restriction
induces an isomorphism $\Pic(X) \cong \Pic(Y)$ if $\dim(X) > 4$. (Use the exponential
sequence and the weak Lefschetz theorem. The assumption on $H^2( , \Z)$ is
superfluous, because Weak Lefschetz actually holds for integral cohomology.)}
\vspace{3mm}

There is a commutative diagram of sheaves on $X$, where the rows are exact, and the columns are induced by
the inclusion $\iota :Y \to X$.
\[
\begin{tikzcd}
0\arrow{r} & \Z \arrow{r}\arrow{d} & \mathcal{O}_X \arrow{r}\arrow{d} & \mathcal{O}_X^{\times} \arrow{r}\arrow{d} & 0 \\
0\arrow{r} & \iota_*\Z \arrow{r} & \iota_*\mathcal{O}_Y \arrow{r} & \iota_*\mathcal{O}_Y^{\times} \arrow{r} & 0
\end{tikzcd}
\]
This induces a commutative diagram between the long exact sequences on cohomology, given by the first 2 rows below.
\[
\begin{tikzcd}
H^1(X,\Z)\arrow{r}\arrow{d} & H^1(X,\mathcal{O}_X)\arrow{r}\arrow{d} & \Pic(X) \arrow{r}\arrow{d} & 
H^2(X,\Z)\arrow{r}\arrow{d} & H^2(X,\mathcal{O}_X)\arrow{d} \\
H^1(X,\iota_*\Z)\arrow{r}\arrow{d} & H^1(X,\iota_*\mathcal{O}_Y)\arrow{r}\arrow{d} & 
H^1(X,\iota_*\mathcal{O}_Y^{\times}) \arrow{r}\arrow{d} &
H^2(\iota_*X,\Z)\arrow{r}\arrow{d} & H^2(X,\iota_*\mathcal{O}_Y)\arrow{d} \\
H^1(Y,\Z)\arrow{r} & H^1(Y,\mathcal{O}_Y)\arrow{r} & \Pic(Y) \arrow{r} & H^2(Y,\Z)\arrow{r} & H^2(Y,\mathcal{O}_Y) \\
\end{tikzcd}
\]
The second and third row are related by natural isomorphisms. Indeed, the inclusion $\iota$ is an affine map,
so $\iota_*$ preserves injective resolutions. In particular, $\iota_*$ induces natural isomorphisms on cohomology.

At this point, we forget about the second row, and consider the commutative diagram formed by the first and third rows.
\[
\begin{tikzcd}
H^1(X,\Z)\arrow{r}\arrow{d}{f_1} & H^1(X,\mathcal{O}_X)\arrow{r}\arrow{d}{f_2} & \Pic(X) \arrow{r}\arrow{d}{f_3} & 
H^2(X,\Z)\arrow{r}\arrow{d}{f_4} & H^2(X,\mathcal{O}_X)\arrow{d}{f_5} \\
H^1(Y,\Z)\arrow{r} & H^1(Y,\mathcal{O}_Y)\arrow{r} & \Pic(Y) \arrow{r} & H^2(Y,\Z)\arrow{r} & H^2(Y,\mathcal{O}_Y) \\
\end{tikzcd}
\]
$f_1, f_2, f_4, f_5$ are isomorphisms due to Weak Lefschetz.\footnote{Weak Lefschetz holds with $\Z$ coefficients, but
the version proved in \cite{Huy} uses $\C$ coefficients. Still, using this version, we obtain isomorphisms
$H^q(X,\Z)/T \to H^q(Y,\Z)/T$. Using universal coefficients, $H^1( , \Z)$ never has torsion, and $H^2( , \Z)$ has no
torsion by hypothesis. Hence $f_1$ and $f_4$ are isomorphisms.} By the 5-lemma, $f_3$ is also an isomorphism.


\subsection*{Exercise 5.2.11 in \cite{Huy}}
\emph{Let $X$ be a projective manifold of dimension $n$ and let $L \in \Pic(X)$ be an
ample line bundle.}
\begin{enumerate}
\item \emph{Show that $m \mapsto h^0(X,L^m)$ for $m >>0$ is a polynomial of degree $n$ with
positive leading coefficient.}
\item \emph{Deduce from this that $a(X) = \dim(X) = n$, i.e. $X$ is Moishezon. Use the
arguments of Section 2.2.}
\end{enumerate}
\vspace{3mm}

\begin{enumerate}
\item If $L$ is ample, then $L$ is positive: for some $m>0$, $L^m = \phi^*\mathcal{O}_{\PP^N}(1)$, so
$c_1(L) = \frac{1}{m} [\omega_{FS}]_X > 0$. Then, by Serre vanishing, $H^q(X,L^m) = 0$ for all $m>m_0$ and
$q>0$. Then, for $m>m_0$, $h^0(X,L^m) = \chi(X,L^m)$, and the function $m \mapsto h^0(X,L^m)$ is the Hilbert
polynomial of the polarized variety $(X,L)$. By exercise 5.1.3, this is a polynomial of degree $n$ and leading
coefficient:
\[ \frac{1}{n!} \int_X c_1(L)^n. \]

\item Define the graded ring:
\[	R(X,L) = \bigoplus_{n\geq 0} H^0(X,L^m) .	\]
$R(X,L)$ is a finite type\todo{any subtlety here? could it be infinitely generated?} algebra over the field $\C$. 
Using Noether normalization, there exists a finite, injective map $\C[x_0, \dots, x_k] \to R(X,L)$. In particular,
the graded pieces $\C[x_0, \dots, x_k]_m$ and $R(X,L)_m$ agree for $m$ large enough. We claim that
$\dim_{\C} \C[x_0, \dots, x_k]_m$ is a polynomial of degree $k$ in $m$, so by part 1 it must be the case that $k=n$. The claim
follows easily in the case that all $x_i$ have degree 1:
\[	\dim_{\C} \C[x_0, \dots, x_k]_m = \binom{m+k}{k} = \frac{(m+k) \dots (m+1)}{k!} .	\]
In general... there should still be an argument? \todo{finish this}

We conclude that:
\[	\trdeg_{\C}	Q(R(X,L)) - 1 = \trdeg_{\C} Q(\C[x_0, \dots, x_n]) - 1 = n. \]
From Proposition 2.2.20 and Proposition 2.1.9, we have the following inequalities:
\[	n = \trdeg_{\C}	Q(R(X,L)) - 1 \leq a(X) \leq n.	\]
Therefore $a(X) = n$, and $X$ is Moishezon.
\end{enumerate}



\subsection*{Exercise 5.3.1 in \cite{Huy}}
\emph{Show that a complex torus $X =\C^2/\Gamma$ is abelian, i.e. projective, if and only if
there exists a line bundle $L$ with $\int c_1^2(L) > 0$. (In fact, this criterion is valid for any
compact complex surface, but more difficult to prove in general.)}
\vspace{3mm}


\subsection*{Exercise 5.3.2 in \cite{Huy}}
\emph{Show that any vector bundle $E$ on a projective manifold $X$ can be written
as a quotient $(L^k)^{\oplus l} \twoheadleftarrow E$ with $L$ an ample line bundle, $k << 0$ and $l>> 0$. (Imitate
the first step of the proof of Proposition 5.3.1, which shows the assertion for $E = \mathcal{O}_X$.
You will also need to revisit Proposition 5.2.7.)}
\vspace{3mm}


\subsection*{Exercise 5.3.3 in \cite{Huy}}
\emph{Let $\sigma : \hat X \to X$ be the blow-up of $X$ in $x \in X$ with the exceptional divisor
$E$ and let $L$ be an ample line bundle on $X$. Show that $\sigma^*L^k \otimes \mathcal{O}(—E)$ is ample for
$k>>0$.
(This is obtained by a revision of the proof of the Kodaira embedding theorem.
In particular, the blow-up $X$ of a projective manifold $X$ is again projective.)}
\vspace{3mm}


\subsection*{Exercise 5.3.4 in \cite{Huy}}
\emph{Continue Exercise 5.1.6 and show, by using the techniques of this section,
that any line bundle $L$ of degree $\deg(L) > 2g$ on a compact curve $C$ of genus $g$ is
very ample, i.e. the linear system associated with $L$ embeds $C$.
Conclude that any elliptic curve is isomorphic to a plane curve, i.e. to a hypersurface
in $\PP^2$.}
\vspace{3mm}

We proceed in 3 steps.
\begin{enumerate}
\item We first show that $L$ is basepoint free. This was already proved in 5.1.6, using the weaker assumption that
$\deg(L) > 2g-1$. Another short proof can be given by using the short exact sequence:
\begin{equation}
\label{eq:morph}
	0 \to L(-p) \to L \to L_p \to 0.
\end{equation}
Then $\deg \omega_C^{-1} \otimes L(-p) > 2g-1 - (2g-2) = 1$, so $L(-p) \otimes \omega_C^{-1}$ is positive. Then Kodaira vanishing
implies that $H^1(L(-p)) = 0$, and the sequence \ref{eq:morph} shows that $H^0(L) \to H^0(L_p)$ is surjective.

\item Since $L$ is basepoint free, it determines a morphism $\phi_L$ to $\PP^n$. We show that $\phi_L$ separates points,
using another short exact sequence.
\begin{equation}
\label{eq:inj}
	0 \to L(-p-q) \to L \to L_{p} \oplus L_{q} \to 0.
\end{equation}
Then $\deg \omega_C^{-1} \otimes L(-p-q) > 2g-2 - (2g-2) = 0$, so $L(-p-q) \otimes \omega_C^{-1}$ is positive. Then Kodaira vanishing
implies that $H^1(L(-p-q)) = 0$, and the sequence \ref{eq:inj} shows that $H^0(L) \to H^0(L_p\oplus L_q)$ is surjective.
In particular, there exist sections of $L$ which vanish at $p$ but not at $q$, and vice-versa. Hence $\phi_L$ is injective.

\item It remains to show that $\phi_L$ separates tangent directions. It is shown in Section 5.3 that injectivity of
the differential $d\phi_L : T_pC \to T_{\phi_L(p)}\PP^n$ is equivalent to surjectivity of $H^0(C,L(-p)) \to H^0(C,
L_p\otimes \omega_{C,p}$. We use the short exact sequence:
\begin{equation}
\label{eq:immer}
	0 \to L(-2p) \to L(-p) \to L_{p} \otimes \omega_{C,p} \to 0.
\end{equation}
Again, $L(-2p) \otimes \omega_C^{-1}$ is positive, so $H^1(C,L(-2p)) = 0$, and the result follows.
\end{enumerate}

In particular, any line bundle $L$ of degree 3 on an eliptic curve gives an embedding into projective space. Using
Riemann-Roch, $h^0(C,L) = 3$, so the morphism $\phi_L$ goes to a projective space of dimension $3-1 = 2$.


\subsection*{Exercise 5.3.5 in \cite{Huy}}
 Show that there exists a complex torus $X$ of dimension $n$ such that $X$ is
projective and, therefore, $a(X) = n$.
\vspace{3mm}

Let $X = \C^n / \Gamma$, where $\Gamma = \langle (1, 0, \dots, 0), (i,0,\dots, 0), \dots, (0, \dots, 1), (0,\dots, i)
\rangle_{\Z}$. We define a Riemann form on $(\C^n, \Gamma)$, which proves that $X$ is an Abelian variety.
\begin{align*}
\omega : \C^n \times \C^n &\to \R, \\
(z,w) &\mapsto \Im(w\bar z).
\end{align*}
It's immediate that $\omega(iz, iw) = \omega(z,w)$ and that $\omega(z, iz) =\Im(i z \bar z) = |z|^2 \geq 0$. Moreover,
$\Gamma$ is chosen so that $\omega(u,v) \in \{-1, 0, 1\}$ for any $u,v$ generators of $\Gamma$. Therefore $\omega$
is a Riemann form, and Corollary 5.3.5 ensures that $\omega$ defines a nonzero class in $H^{1,1}(X,\Z)$. Hence
$X$ is projective.

The last statement follows from Exercise 3.2.11, where it is proved that projective manifolds are Moishezon.



\section{Voisin, Volume I}
\subsection*{Exercise 6.2 in \cite{Voi1}}
\emph{The Hodge decomposition for curves. Let $X$ be a compact connected complex curve. We have the differential:}
\[	d : \mathcal{O}_X \to \Omega_X	\]
\emph{between the sheaf of holomorphic functions and the sheaf of holomorphic differentials.}
\begin{enumerate}[(a)]
\item \emph{Show that $d$ is surjective with kernel equal to the constant sheaf $\C$. Hence we have an exact
sequence:}
\begin{equation}
\label{eq:dcurve}
 0 \to \C \to \mathcal{O}_X \overset{d}{\to} \Omega_X \to 0 .
\end{equation}
\item \emph{Deduce from Serre duality that $H^1(X, \Omega_X) \cong \C$. Deduce from Poincar\'e duality that
$H^2(X, \C) = \C$.}
\item \emph{Show that \ref{eq:dcurve} induces a short exact sequence:}
\begin{equation}
\label{eq:hodgecurve}
0 \to H^0(X, \Omega_X) \to H^1(X, \C) \to H^1(X, \mathcal{O}_X) \to 0.
\end{equation}
\item \emph{Show that the map which to a holomorphic form $\alpha$ associates the class of $\bar \alpha$ in
$H^1(X, \mathcal{O}_X)$ is injective.}
\item \emph{Deduce from Serre duality that it is also surjective and that we have the decomposition:}
\begin{equation}
\label{eq:decompcurve}
H^1(X, \C) = H^0(X,\Omega_X) \oplus \overline{H^0(X,\Omega_X)},
\end{equation}
\emph{with}
\[	\overline{H^0(X,\Omega_X)} \cong H^1(X, \mathcal{O}_X).	\]
\end{enumerate}

\begin{proof}
\text{   }
\begin{enumerate}[(a)]
\item We check exactness at the level of stalks. On small enough contractible open sets $U$, we use 
the Poincar\'e Lemma to express any 1-form
$\omega \in \mathcal{A}^1(U)$ as $df$ for some $f \in C^{\infty}(U)$. Then we have
$\omega = \p_z f dz + \p_{\bar z} f d\bar z$ in local coordinates on $U$. In the case $\omega$ holomorphic,
its coordinate representation in any chart must be holomorphic, which gives $\p_{\bar z} f = 0$, i.e. $f \in
\mathcal{O}_X(U)$. This construction works for any contractible $U$ and is compatible with restriction,
so we obtain surjectivity on the inverse limits $\mathcal{O}_{X,x} \to \Omega_{X,x}$, for all $x$. Moreover,
for all $U$, $df = 0$ iff $f$ is constant, which gives:
\[	 0 \to \C_x \to \mathcal{O}_{X,x} \overset{d}{\to} \Omega_{X,x} \to 0.	\]
This is equivalent to \ref{eq:dcurve}.

\item Serre duality, together with the fact that $\Omega_X = K_X$ for curves, gives:
\[	H^1(X,\Omega_X) = H^1(X, \mathcal{O}_X \otimes \Omega_X) = H^1(X, \mathcal{O}_X) \cong \C.	\]
From Poincar\'e duality, $H^2(X,\C) \cong H^0(X,\C) = \C$. This also follows from the fact that $X$ is orientable.

\item We have a long exact sequence associated to \ref{eq:dcurve}. The map:
\[	H^0(X, \mathcal{O}_X) \overset{d}{\to} H^0(X,\Omega_X)	\]
is 0, because the compactness of $X$ implies that all elements of $H^0(X, \mathcal{O}_X)$ are constants.
Hence the following is exact:
\begin{equation}
\label{eq:longasscurve}
0 \to H^0(X,\Omega_X) \to H^1(X, \C) \to H^1(X,\mathcal{O}_X) \to H^1(X,\Omega_X) \to H^2(X, \C) \to
H^2(X,\mathcal{O}_X).
\end{equation}
Using the Dolbeault resolution of $\mathcal{O}_X$,
\[	H^2(X,\mathcal{O}_X) \cong \frac{\Ker(\bar \p : \mathcal{A}^{0,2} \to \mathcal{A}^{0,3})}{\Imag
(\bar \p : \mathcal{A}^{0,1} \to \mathcal{A}^{0,2})} = 0, \]
because $\mathcal{A}^{0,2} = 0$ on a curve. Thus, the last term of \ref{eq:longasscurve} is 0, so the map
$H^1(X,\Omega_X) \to H^2(X, \C)$ must be surjective. From part (b), both these terms are $\C$, so the map
is actually an isomorphism. This means we can split \ref{eq:longasscurve} even further to get \ref{eq:hodgecurve}.

\item Using the Dolbeault resolution of $\mathcal{O}_X$ again, it makes sense to think of $\bar \alpha$ as an
element of $H^1(X, \mathcal{O}_X)$. If $\alpha$ is $d$-closed, $\bar \alpha$ is $\bar \p$-closed, and if
$\alpha$ is $d$-exact, $\bar \alpha$ is $\bar \p$-exact.\footnote{Because $d = \p + \bar \p$, in general,
so $d=\p$ on holomorphic forms.} Therefore we have a map:
\[	\psi: H^0(X,\Omega_X) \to H^1(X, \mathcal{O}_X).	\]
Injectivity is just the statement that, if $\bar \alpha = \bar \p \phi$ for some $\phi \in C^{\infty}(X)$, then
$\alpha = d f$ for some $f \in \Gamma(\mathcal{O}_X)$. This follows from the fact that $\alpha = \p \overline{\phi}$,
but $\alpha$ is assumed holomorphic, so $\overline{\phi}$ is also holomorphic. In this case, $\p \overline{\phi}
= d \overline{\phi}$.

\item Using Serre duality again, there is a canonical isomorphism $H^1(X, \mathcal{O}_X) \cong 
\big(H^0(X, \Omega_X)  \big)^*$. In particular, $\dim_{\C} H^1(X, \mathcal{O}_X) = \dim_{\C} H^0(X, \Omega_X)$,
so the map $\psi$ in part (d) is an isomorphism for dimension reasons. This means we have the following
isomorphisms:
\[	H^0(X,\Omega_X) \overset{\psi}{\to} H^1(X, \mathcal{O}_X) \overset{\cong}{\to} \big(H^0(X, \Omega_X)  \big)^*.	\]
Since $\psi$ is simply complex conjugation, we obtain a canonical isomorphism:
\[	\overline{H^0(X,\Omega_X)} \overset{\cong}{\to} H^1(X, \mathcal{O}_X).	\]
We replace $H^1(X, \mathcal{O}_X)$ in the short exact sequence \ref{eq:hodgecurve}:
\[	0 \to H^0(X, \Omega_X) \to H^1(X, \C) \to \overline{H^0(X,\Omega_X)} \to 0.	\]
It remains to show that this sequence splits.


\todo{finish}
\end{enumerate}
\end{proof}



\subsection*{Exercise 8.1 in \cite{Voi1}}
\emph{Let $S$ be a compact complex surface.}
\begin{enumerate}[(a)]
\item \emph{Show that the Fr\"olicher spectral sequence of $S$ degenerates at $E_3$ for degree reasons. What are
the possibly non-zero differentials $d_2$?}
\item \emph{By studying the integrals:}
\[	\int_S \alpha \wedge \bar \alpha	\]
\emph{for $\alpha$ a holomorphic form of degree 2, show that the holomorphic 1-forms on $S$ are closed.}
\item \emph{More generally, show that the map}
\[	H^0(S,\Omega^2_S) \to H^2(S,\C)	\]
\emph{which to a holomorphic 2-form associates its de Rham class, is injective. Deduce from this that the 
differential}
\[	d_2^{0,1} : E_2^{0,1} \to E_{2}^{2,0}	\]
\emph{vanishes. (Notice that the last space is equal to $H^0(S, \Omega^2_S)$ by part (b).}
\item \emph{Show that the differentials}
\[	d_1 : H^q(S,\Omega_S^p) \to H^q(S,\Omega_S^{p+1})	\]
\emph{and}
\[	d_1 : H^{2-q}(S,\Omega_S^{2-p-1}) \to H^{2-q}(S,\Omega_S^{2-p})	\]
\emph{are dual up to sign with respect to Serre duality. Deduce from this and (b) that}
\[	d_1 = \p : H^2(S,\mathcal{O}_S) \to H^2(S,\Omega_S)	\]
\emph{is equal to zero.}
\item \emph{Show that the map}
\[	H^0(S,\Omega_S) \to H^1(S,\C)	\]
\emph{which to a holomorphic 1-form (note that this is closed by part (b)) associates its de Rham class, is injective.}
\item \emph{Let $\delta := \p : H^1(S,\mathcal{O}_S) \to H^1(S,\Omega_S)$. Show that we have the relations:}
\begin{align*}
b_1(S) &= h^{1,0}(S) + h^{0,1}(S) - \rk(\delta), \\
b_3(S) &\leq h^{2,1}(S) + h^{1,2}(S) - \rk(\delta),
\end{align*}
\emph{with equality if and only if $d_2^{0,2} = 0$. Deduce from this that the Fr\"olicher spectral sequence
of $S$ degenerates at $E_2$ and that it degenerates at $E_1$ if and only if $\delta = 0$.}
\item \emph{Show that we have the relations:}
\begin{align*}
b_3(S) &= b_1(S) = h^{1,0}(S) + h^{0,1}(S) - \rk(\delta), \\
b_2(S) &\leq h^{2,0}(S) + h^{0,2}(S) + h^{1,1}(S) - 2\rk(\delta).
\end{align*}
\emph{NB: One can show that $\delta$ is in fact always 0, that is the Fr\"olicher spectral sequence of a compact
complex surface degenerates at $E_1$.}
\end{enumerate}


\begin{proof}
Note that $S$ is not assumed K\"ahler in this problem.
\begin{enumerate}[(a)]
\item By definition, $E_r^{p,q}$ are subquotients of $F^p\mathcal{A}^{p+q} = \mathcal{A}^{p,q} \oplus \dots
\oplus \mathcal{A}^{p+q,0}$. On a complex surface $S$, the only such sheaves which can be nonzero are:
\[
\begin{tikzcd}
F^0\mathcal{A}^4 &  &  \\
F^0\mathcal{A}^3 & F^1\mathcal{A}^3 & \\
F^0\mathcal{A}^2 & F^1\mathcal{A}^2 & F^2\mathcal{A}^2 \\
F^0\mathcal{A}^1 & F^1\mathcal{A}^1 & F^2\mathcal{A}^1 \\
F^0\mathcal{A}^0 & F^1\mathcal{A}^0 & F^2\mathcal{A}^0 \\
\end{tikzcd}
\]
The differential $d_3$ has degree (3,-2), so either its domain or its target is 0. Therefore the sequence
degenerates at $E_3$. The morphisms $d_2$, with degree (2,-1), allowed to be nonzero are the ones 
with target in the rightmost
column of the diagram above, i.e. $d_2^{0,1}$, $d_2^{0,2}$ and $d_2^{0,3}$.

\item If $\beta$ is holomorphic of degree 1, $d\beta = \p \beta$, so $\overline{d\alpha} = \bar \p \bar \beta$,
and $\bar \beta$ is antiholomorphic. In particular, we have:
\[  d(\beta \wedge \bar \p \bar \beta) = \p \beta \wedge \bar \p \bar \beta - \beta \wedge \bar \p^2 \bar \beta
= d\beta \wedge \overline{d\beta}. \]
Now, since $d\beta \wedge \overline{d\beta}$ is exact, Stokes' theorem implies that
\[	\int_S d\beta \wedge \overline{d\beta} = 0.	\]
However, writing $d\beta = f(z) dz_1 \wedge dz_2$ locally, the above gives:
\[	\int_S |f(z)|^2 dz_1 \wedge dz_2 \wedge d\bar z_1 \wedge d\bar z_2 = 0.	\]
This is only possible if $f = 0$, which implies $d \beta = 0$. We conclude that all holomorphic 1-forms on $S$
are closed.

\item If $\omega \in H^0(S,\Omega^2_S)$, then $d\omega = 0$, since $S$ cannot support holomorphic 3-forms. This
gives a map $H^0(S,\Omega^2_S) \to H^2(S,\C)$. If $\omega = d \beta$ with $\beta \in \mathcal{A}^1(S)$, then
$\beta$ can be chosen holomorphic. Indeed, in local coordinates, is $\beta = f(z_i) dz_i + g(z_j) d\bar z_j$,
then $d \beta$ holomorphic implies that $f$ is holomorphic and $g$ is constant, and we may replace $\beta$ by
$f(z_i) dz_i$. However, by part (b), $\beta \in \Omega_S^1$ satisfies $d\beta = 0$, so $\omega = 0$. This
proves that $H^0(S,\Omega^2_S) \to H^2(S,\C)$ is injective.

The nonzero part of the $E_1$ page of the spectral sequence is:
\[
\begin{tikzcd}
H^2(S,\mathcal{O}_S)\arrow[dashed]{r} & H^2(S,\Omega^1_S)\arrow[dotted]{r} & H^2(S,\Omega^2_S) \\
H^1(S,\mathcal{O}_S)\arrow{r} & H^1(S,\Omega^1_S)\arrow{r} & H^1(S,\Omega^2_S) \\
H^0(S,\mathcal{O}_S)\arrow[dotted]{r} & H^0(S,\Omega^1_S)\arrow[dashed]{r} & H^0(S,\Omega^2_S).
\end{tikzcd}
\]
Compare this to the diagram in part (a): the groups $H^4(S,\mathcal{O}_S)$, $H^3(S,\mathcal{O}_S)$ and
$H^3(S,\Omega^1_S)$ are zero, because they can be expressed in terms of $\mathcal{A}^{0,4}$, $\mathcal{A}^{0,3}$
and $\mathcal{A}^{1,3}$ forms respectively, using the Dolbeault resolutions of $\mathcal{O}_S$ and $\Omega^1_S$.

The injectivity of $H^0(S,\Omega^2_S) \to H^2(S,\C)$ means that $H^0(S,\Omega^2_S)$ survives to the $E_{\infty}$
page, so all differentials with target $H^0(S,\Omega^2_S)$ are zero. Thus, $d_1^{1,0} = 0$ and $d_2^{0,1} = 0$.

\item Since $K_S = \Omega_S^2$, the pairs of groups situated opposite to each other with respect to the center of
the diagram in part (c) are dual to each other, by Serre duality. Using the functoriality of Serre duality,
the pairs drawn as solid, dashed and dotted represent dual maps. In particular, in part (c) we showed that
the lower right dashed arrow $d_1^{1,0} = 0$. This means that $d_1^{0,2} : H^2(S,\mathcal{O}_S) \to
H^2(S,\Omega^1_S)$ is also zero.

\item In exercise 6.2 in \cite{Voi1}, solved above, we saw that the short exact sequence of sheaves:
\[	0 \to \C \to \mathcal{O}_S \to \Omega^1_S \to 0	\]
induces a long exact sequence in cohomology, part of which is:
\[	0 \to H^0(S,\Omega^1_S) \to H^1(S,\C) \to \dots	\]
So the map $H^0(S,\Omega^1_S) \to H^1(S,\C)$ is injective. This is just expressing the fact that $H^0(S,\mathcal{O}_S)
\cong \C)$.

\item From part (e), we deduce that the group $H^0(S,\Omega^1_S)$ survives to the $E_{\infty}$ page, so
$d_1^{0,0} = 0$. Using part (d), both dotted arrows in the diagram are 0, and we are left with:
\[
\begin{tikzcd}
H^2(S,\mathcal{O}_S)\arrow[dashed]{r}{0} & H^2(S,\Omega^1_S)\arrow[dotted]{r}{0} & H^2(S,\Omega^2_S) \\
H^1(S,\mathcal{O}_S)\arrow{r}{\delta} & H^1(S,\Omega^1_S)\arrow{r}{\delta^*} & H^1(S,\Omega^2_S) \\
H^0(S,\mathcal{O}_S)\arrow[dotted]{r}{0} & H^0(S,\Omega^1_S)\arrow[dashed]{r}{0} & H^0(S,\Omega^2_S).
\end{tikzcd}
\]
Moreover, recall from part (c) that $d_2^{0,1} = 0$, so the only arrow on $E_2$ which is possibly nonzero
is $d_2^{0,2}$, which does not affect the groups $H^1(S,\mathcal{O}_S)$ and $H^0(S,\Omega^1_S)$ which contribute
to $\Gr H^1(S,\C)$. Therefore:
\[	b_1(S) = h^{1,0}(S) + \dim \Ker \delta = h^{1,0}(S) + h^{0,1}(S) - \rk \delta.	\]
The differential $d_2^{0,2}$ could affect the groups contributing to $\Gr H^3(S,\C)$, so we have:
\[	b_3(S) = h^{1,2}(S) + h^{2,1}(S) - \rk \delta^* - \rk d_2^{0,2} \leq h^{1,2}(S) + h^{2,1}(S) - \rk \delta ,	\]
with equality if and only if $d_2^{0,2} = 0$.

By Poincar\'e duality, $b_1(S) = b_3(S)$, which shows that $d_2^{0,2} = 0$ and the spectral sequence degenerates
at $E_2$.

\item From part (f) we obtain immediately that:
\begin{align*}
b_1(S) = b_3(S) &= h^{1,0}(S) + h^{0,1}(S) - \rk \delta, \\
b_2(S) &= h^{2,0}(S) + h^{0,2}(S) + \dim \Ker \delta^* - \dim \Imag \delta = h^{2,0}(S) + h^{0,2}(S) +
h^{1,1} - 2 \rk \delta.
\end{align*}
\end{enumerate}
\end{proof}



\subsection*{Exercise 8.2 in \cite{Voi1}}
\emph{Spherical spectral sequences. Let $(M^{\bullet},F)$ be a filtered complex in an abelian category. We assume
that the decreasing filtration $F$ satisfies the finiteness condition $F^pM^k = 0$ for $p$ large enough. We
assume that there exist two integers $p<p'$ such that the complexes $\Gr^k M^{\bullet}$ are exact for $k\neq p,p'$.}
\begin{enumerate}[(a)]
\item \emph{Show that the differentials $d_i$, $i\geq 1$ of the spectral sequence of $(M^{\bullet},F)$ are 0
for $i\neq p'-p =:r$.}

\item \emph{Show that $E_{r+1}^{p,q} = E_{\infty}^{p,q}$ is equal to 0 for $k\neq q,q'$ and that one has an
exact sequence:}
\[	0 \to E^{p',q}_{\infty} \to H^{p'+q}(M^{\bullet}) \to E^{p,q+p'-p}_{\infty} \to 0.	\]

\item \emph{Show that}
\begin{align*}
E_{\infty}^{p',q} &= E_r^{p',q}/\Imag d_r = E_1^{p',q}/\Imag d_r, \\
E_{\infty}^{p,q} &= \Ker d_r \subset E_r^{p,q} = E_1^{p,q}.
\end{align*}

\item \emph{Deduce that one has a long exact sequence:}
\[	\cdots \to H^k(M^{\bullet}) \to E_1^{p,k-p} \overset{d_r}{\to} E_1^{p',k-p'+1} \to 
H^{k+1}(M^{\bullet}) \to \cdots	\]
\emph{where $E_1^{p,k-p} = H^k(\Gr^p M^{\bullet})$ and $E_1^{p',k-p'} = H^k(\Gr^{p'} M^{\bullet})$.}

\emph{These spectral sequences appear as Leray spectral sequences of the sphere bundles.}
\end{enumerate}


\begin{proof}
\text{  }
\begin{enumerate}[(a)]
\item The $E_0$ page of the spectral sequence is:
\[
\begin{tikzcd}
\vdots & \vdots & \vdots \\
\Gr^0 M^2\arrow{u} & \Gr^1 M^3\arrow{u} & \Gr^2 M^4\arrow{u} \\
\Gr^0 M^1\arrow{u} & \Gr^1 M^2\arrow{u} & \Gr^2 M^3\arrow{u} \\
\Gr^0 M^0\arrow{u} & \Gr^1 M^1\arrow{u} & \Gr^2 M^2\arrow{u} \\
\Gr^0 M^{-1}\arrow{u} & \Gr^1 M^0\arrow{u} & \Gr^2 M^1\arrow{u} \\
\vdots\arrow{u} & \vdots\arrow{u} & \vdots\arrow{u} 
\end{tikzcd}
\]
The double complex is possibly unbounded above and below, but it is bounded to the left and eventually bounded 
to the right. By assumption, all columns except for $p$ and $p'$ are exact. Therefore the $E_1$ page has nonzero
groups only in the $p$ and $p'$ columns. Since $E_r^{p,q}$ is a subquotient of $E_1^{p,q}$ for $r>1$, this is true
for all subsequent pages. Now $\deg d_r = (r,-r+1)$, so the only differentials which are possibly nonzero are
$d_r$ for $r = p'-p$.

\item From part (a) it follows that $E_1 = \dots = E_r$, and $E_{r+1} = \dots = E_{\infty}$. Moreover, since all
nonzero groups are in the columns $p$ and $p'$, the following hold for all $q$:
\[	H^{p+q}(M^{\bullet}) = F^0 H^{p+q}(M^{\bullet}) = \dots = F^p H^{p+q}(M^{\bullet}), 	\]
\[	F^{p+1} H^{p+q}(M^{\bullet}) = \dots = F^{p'} H^{p+q}(M^{\bullet}), \]
\[	F^{p'+1} H^{p+q}(M^{\bullet}) = 0.	\]
Therefore:
\[	E_{\infty}^{p',q} = \Gr^{p'} H^{p'+q}(M^{\bullet}),	\]
\[	E_{\infty}^{p,q+r} = F^p H^{p+q+r}(M^{\bullet})	/ F^{p+1} H^{p+q+r}(M^{\bullet}) = 
H^{p+q+r}(M^{\bullet}) / E_{\infty}^{p+r,q}. \]
Putting these together gives the exact sequence:
\[	0 \to E_{\infty}^{p',q} \to H^{p+q+r}(M^{\bullet}) \to E_{\infty}^{p,q+p'-p} \to 0. 	\]

\item The equalities are immediate from $E_{r+1}^{p,q} = \Ker \big(d_r^{p,q} : E_r^{p,q} \to E_r^{p+r,q-r+1}\big)
/ \Imag \big(d_r^{p-r,q+r-1} : E_r^{p-r,q+r-1} \to E_r^{p,q}\big)$.

\item Part (c) gives the exact sequence:
\[	0 \to E_{\infty}^{p,q} \to E_{1}^{p,q} \overset{d_r}{\to} \to E_{1}^{p+r,q-r+1} \to E_{1}^{p+r,q-r+1} \to 0. 	\]
While from part (b) we have:
\[	0 \to E_{\infty}^{p+r,q} \to H^{p+r+q}(M^{\bullet}) \to E_{\infty}^{p,q+r} \to 0.	\]
We concatenate these, increasing $q$ by 1 each time. Concatenation makes the $E_{\infty}$ groups disappear,
and letting $k= p+r+q$ gives:
\[	\cdots \to H^k(M^{\bullet}) \to E_1^{p,k-p} \to E_1^{p',k-p'+1} \to H^{k+1}(M^{\bullet}) \to \cdots	\]

\end{enumerate}
\end{proof}




\bibliography{./cls/bib}{}
\bibliographystyle{alpha}

\end{document}









